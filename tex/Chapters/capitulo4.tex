\chapter{ESPACIOS CON PRODUCTO INTERNO}
\printchaptertableofcontents

Si $\mathbf{x} = (x_1, x_2, \dots, x_n)$ e $\mathbf{y} = (y_1, y_2, \dots, y_n)$ son puntos de $\RR[n]$, entonces el producto interno, punto o escalar de $\mathbf{x}$ e $\mathbf{y}$ es
$$\mathbf{x} \cdot \mathbf{y} = x_1y_1 + x_2y_2 + \cdots + x_ny_n.$$

En el presente capítulo, este producto, que en general llamamos producto interno, se generalizará con el fin de poderlo aplicar a un espacio lineal abstracto $X$. Un producto interno será, pues, una función de valores escalares definida en $X \times X$, que posea determinadas propiedades. Si tenemos un espacio lineal con un producto interno dado definido en él, lo llamamos espacio con producto interno.

En un espacio con producto interno, la norma puede definirse en términos del producto interno definido. Por consiguiente, un espacio con producto interno es un espacio lineal normado, pero también es algo más; por ejemplo, en un espacio con producto interno puede definirse el concepto de ortogonalidad de vectores, lo cual tiene importantes consecuencias.

La ortogonalidad es crucial, ya que permite, entre otras cosas, la construcción de bases ortogonales o incluso ortonormales, lo cual simplifica muchos problemas, como la resolución de sistemas de ecuaciones o la aproximación de funciones. Además, la ortogonalidad está relacionada con la proyección de vectores, una operación fundamental en el estudio de subespacios y en métodos como la descomposición en valores singulares o la factorización QR. Así, un espacio con producto interno no solo amplía la estructura de un espacio normado, sino que también abre la puerta a herramientas poderosas para analizar y manipular vectores y subespacios.

\newpage

\section{Espacios con producto interno y proyecciones}

\begin{definicion}{}{}
    Sea $X$ un espacio lineal. Definimos un \emph{producto interno} en $X$ como una función $\langle \, , \rangle : X \times X \longrightarrow K$ tal que si $\mathbf{x}$, $\mathbf{y}$, $\mathbf{z} \in X$ y $\alpha \in K$, entonces
    \begin{enumerate}[label=\roman*), topsep=6pt, itemsep=0pt]
        \item $\langle \mathbf{x}, \mathbf{x} \rangle > 0$ si y solo si $\mathbf{x} \neq \mathbf{0}$. \hfill (Positivo definido)
        \item $\langle \mathbf{x} + \mathbf{y}, \mathbf{z} \rangle = \langle \mathbf{x}, \mathbf{z} \rangle + \langle \mathbf{y}, \mathbf{z} \rangle$. \hfill (Aditivo)
        \item $\langle \mathbf{x}, \mathbf{y} \rangle = \overline{\langle \mathbf{y}, \mathbf{x} \rangle}$. \hfill (Hermítico)
        \item $\langle \alpha \mathbf{x}, \mathbf{y} \rangle = \alpha \langle \mathbf{x}, \mathbf{y} \rangle$. \hfill (Homogéneo)
    \end{enumerate}
\end{definicion}

\marginElement{\vspace{3cm}\infoBulle{Si el espacio lineal es real, es decir, los escalares son números reales, entonces la propiedad hermítica se reduce a la simetría, es decir, $\langle \mathbf{x}, \mathbf{y} \rangle = \langle \mathbf{y}, \mathbf{x} \rangle$.}}

\begin{prop}{}{}
    Sea $X$ un espacio lineal y $\langle \, , \rangle$ un producto interno en $X$, entonces:
    \begin{enumerate}[label=\roman*), topsep=6pt, itemsep=0pt]
        \item El producto interno es homogéneo conjugado en su segunda componente. Es decir, para todo $\mathbf{x}$, $\mathbf{y} \in X$ y $\alpha \in K$,
        $$\langle \mathbf{x}, \alpha \mathbf{y} \rangle = \overline{\alpha} \langle \mathbf{x}, \mathbf{y} \rangle.$$
        \item El producto interno es aditivo en su segunda componente. Es decir, para todo $\mathbf{x}$, $\mathbf{y}$, $\mathbf{z} \in X$,
        $$\langle \mathbf{x}, \mathbf{y} + \mathbf{z} \rangle = \langle \mathbf{x}, \mathbf{y} \rangle + \langle \mathbf{x}, \mathbf{z} \rangle.$$
        \item El producto interno es lineal en su primera componente. Es decir, para todo $\mathbf{x}$, $\mathbf{y}$, $\mathbf{z} \in X$ y $\alpha$, $\beta \in K$,
        $$\langle \alpha \mathbf{x} + \beta \mathbf{y}, \mathbf{z} \rangle = \alpha \langle \mathbf{x}, \mathbf{z} \rangle + \beta \langle \mathbf{y}, \mathbf{z} \rangle.$$
        \item El producto interno es lineal en su segunda componente. Es decir, para todo $\mathbf{x}$, $\mathbf{y}$, $\mathbf{z} \in X$ y $\alpha$, $\beta \in K$,
        $$\langle \mathbf{x}, \alpha \mathbf{y} + \beta \mathbf{z} \rangle = \overline{\alpha} \langle \mathbf{x}, \mathbf{y} \rangle + \overline{\beta} \langle \mathbf{x}, \mathbf{z} \rangle.$$
    \end{enumerate}
    
    \tcblower
    \demostracion
    \begin{enumerate}[label=\roman*), topsep=6pt, itemsep=0pt]
        \item Sean $\mathbf{x}$, $\mathbf{y} \in X$ y $\alpha \in K$, entonces
        \begin{align*}
            \langle \mathbf{x}, \alpha \mathbf{y} \rangle & = \overline{\langle \alpha \mathbf{y}, \mathbf{x} \rangle} \\
            & = \overline{\alpha \langle \mathbf{y}, \mathbf{x} \rangle} \\
            & = \overline{\alpha} \overline{\langle \mathbf{y}, \mathbf{x} \rangle} \\
            & = \overline{\alpha} \overline{\overline{\langle \mathbf{x}, \mathbf{y} \rangle}} \\
            & = \overline{\alpha} \langle \mathbf{x}, \mathbf{y} \rangle
        \end{align*}
        \item Sean $\mathbf{x}$, $\mathbf{y}$, $\mathbf{z} \in X$, entonces
        \begin{align*}
            \langle \mathbf{x}, \mathbf{y} + \mathbf{z} \rangle & = \overline{\langle \mathbf{y} + \mathbf{z}, \mathbf{x} \rangle} \\
            & = \overline{\langle \mathbf{y}, \mathbf{x} \rangle + \langle \mathbf{z}, \mathbf{x} \rangle} \\
            & = \overline{\langle \mathbf{y}, \mathbf{x} \rangle} + \overline{\langle \mathbf{z}, \mathbf{x} \rangle} \\
            & = \overline{\overline{\langle \mathbf{x}, \mathbf{y} \rangle}} + \overline{\overline{\langle \mathbf{x}, \mathbf{z} \rangle}} \\
            & = \langle \mathbf{x}, \mathbf{y} \rangle + \langle \mathbf{x}, \mathbf{z} \rangle
        \end{align*}
        \item Sean $\mathbf{x}$, $\mathbf{y}$, $\mathbf{z} \in X$ y $\alpha$, $\beta \in K$, entonces
        \begin{align*}
            \langle \alpha \mathbf{x} + \beta \mathbf{y}, \mathbf{z} \rangle & = \langle \alpha \mathbf{x}, \mathbf{z} \rangle + \langle \beta \mathbf{y}, \mathbf{z} \rangle \\
            & = \alpha \langle \mathbf{x}, \mathbf{z} \rangle + \beta \langle \mathbf{y}, \mathbf{z} \rangle
        \end{align*}
        \item Se demuestra de manera similar al anterior inciso.
    \end{enumerate}
\end{prop}