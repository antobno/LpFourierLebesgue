\chapter{RESPUESTAS}

\noindent\begin{adjustwidth}{-0.75\marginparsep - \marginparwidth}{0mm}
\begin{multicols}{2}

\sectionbox{Capítulo 1}
\addcontentsline{toc}{section}{Capítulo 1}

\begin{enumerate}
    \item \begin{enumerate}
        \item \textbf{Cerradura:} Para cualquier par de enteros $a$ y $b$, la suma $a + b$ es también un número entero. Esto es cierto porque la suma de dos enteros siempre es un entero.
        \item \textbf{Asociatividad:} Para cualesquiera tres enteros $a$, $b$ y $c$, se cumple que
        $$(a + b) + c = a + (b + c).$$
        La adición de números enteros es una operación asociativa.
        \item \textbf{Elemento identidad:} Existe un elemento $e = 0$ tal que para cualquier entero $a$, se cumple que:
        $$a + 0 = 0 + a = a.$$
        El cero actúa como el elemento identidad para la adición.
        \item \textbf{Inverso:} Para cada entero $a$, existe un inverso aditivo $-a$ tal que:
        $$a + (-a) = (-a) + a = 0.$$
        El inverso aditivo de un número entero es simplemente su opuesto.
        \item \textbf{Conmutatividad:} Para cualesquiera $a$ y $b$, se cumple que:
        $$a + b = b + a.$$
    \end{enumerate}
    Dado que $(\ZZ, +)$ cumple con todas estas propiedades, concluimos que es un grupo abeliano.
    \item \begin{enumerate}
        \item \textbf{Cerradura:} Para todo $z_1$, $z_2 \in \CC - \{0\}$, la multiplicación $z_1 \cdot z_2$ es otro número complejo no nulo. Esto se debe a que la multiplicación de dos números complejos siempre da como resultado otro número complejo, y si ninguno de los dos números es cero, el producto tampoco lo será.
        \item \textbf{Asociatividad:} Para todo $z_1$, $z_2$, $z_3 \in \CC - \{0\}$,
        $$(z_1 \cdot z_2) \cdot z_3 = z_1 \cdot (z_2 \cdot z_3).$$
        La propiedad de asociatividad se hereda de la multiplicación de números reales, que es asociativa, y la multiplicación de números complejos se define en términos de números reales.
        \item \textbf{Elemento identidad:} Existe $1 \in \CC - \{0\}$ tal que para todo $z \in \CC - \{0\}$,
        $$z \cdot 1 = 1 \cdot z = z.$$
        El número complejo $1$ actúa como el elemento identidad para la multiplicación de números complejos.
    \end{enumerate}
\end{enumerate}

\end{multicols}
\end{adjustwidth}

\newpage

\respuestasPage
\twocolumn

\sectionbox{Capítulo 2}
\addcontentsline{toc}{section}{Capítulo 2}

b

\newpage
\,

\newpage

%\respuestasPage
%\twocolumn
%\begin{multicols}{2}

\sectionbox{Capítulo 3}
\addcontentsline{toc}{section}{Capítulo 3}

\begin{enumerate}
    \item Sea $\mathbf{x} = (x_1, x_2, \dots, x_n) \in \RR[n]$.
    \begin{enumerate}
        \item Como cada sumando es no negativo, en particular se cumple que
        $$x_i^2 \leq x_1^2 + x_2^2 + \dots + x_n^2.$$
        Tomando raíz cuadrada en ambos lados
        $$|x_i| = \sqrt{x_i^2} \leq \sqrt{x_1^2 + x_2^2 + \dots + x_n^2} = \| \mathbf{x} \|.$$
        Ahora bien, sea $M = \max \{|x_1|, |x_2|, \dots, |x_n|\}$. Como $|x_i| \leq M$ para todo $i$, se sigue que
        $$x_j^2 \leq M^2.$$
        Sumando sobre todas las componentes,
        $$x_1^2 + x_2^2 + \cdots + x_n^2 \leq n \cdot M^2.$$
        Tomando raíz cuadrada,
        \begin{align*}
            \| \mathbf{x} \| & = \sqrt{x_1^2 + x_2^2 + \cdots + x_n^2} \\
            & \leq \sqrt{n \cdot M^2} \\
            & = \sqrt{n} \cdot M
        \end{align*}
        De esta forma,
        $$|x_i| \leq \| \mathbf{x} \| \leq \sqrt{n} \max \{ |x_1|, |x_2|, \dots, |x_n|\}.$$
        \item Ya demostramos que $|x_i| \leq \| \mathbf{x} \|$ para todo $i$. Como la norma infinito es el máximo de todos los $|x_i|$, se sigue que
        $$\max \{ |x_1|, |x_2|, \dots, |x_n| \} \leq \| \mathbf{x} \|.$$
        Es decir,
        $$\| \mathbf{x} \|_{\infty} \leq \| \mathbf{x} \|.$$
        Ya demostramos que
        $$\| \mathbf{x} \| \leq \sqrt{n} \max \{ |x_1|, |x_2|, \dots, |x_n|\},$$
        pero recordemos que
        $$\max \{|x_1|, |x_2|, \dots, |x_n|\} = \| \mathbf{x} \|_{\infty},$$
        así que simplemente reemplazamos
        $$\| \mathbf{x} \| \leq \sqrt{n} \| \mathbf{x} \|_{\infty}.$$
        De esta forma,
        $$\| \mathbf{x} \|_{\infty} \leq \| \mathbf{x} \| \leq \sqrt{n} \| \mathbf{x} \|_{\infty}.$$
    \end{enumerate}
    \item \begin{enumerate}
        \item Para graficar la bola, es necesario observar que
        \begin{align*}
            (x, y) \in B(\mathbf{0}, 1) & \Longleftrightarrow \| (x, y) - (0, 0) \|_1 \leq 1 \\
            & \Longleftrightarrow \| (x, y) \|_1 \leq 1 \\
            & \Longleftrightarrow |x| + |y| \leq 1
        \end{align*}
        La desigualdad $|x| + |y| < 1$ describe un rombo (o diamante) centrado en el origen $(0, 0)$ con vértices en los puntos: $(1, 0)$, $(-1, 0)$, $(0, 1)$, $(0, -1)$. Cada arista del rombo corresponde a una ecuación lineal: $x + y = 1$, $-x + y = 1$, $-x - y = 1$, $x - y = 1$ para cada cuadrante respectivamente. Por lo tanto, la representación gráfica de esta bola es:
        \begin{center}
            \begin{tikzpicture}
                \draw[-stealth] (-2.5,0) -- (2.5,0) node[below left] {$x$};
                \draw[-stealth] (0,-2.5) -- (0,2.5) node[below left] {$y$};
                \filldraw[cw2,opacity=0.1] (0,1) -- (-1,0) -- (0,-1) -- (1,0) -- cycle;
                \draw[cw0] (0,1) -- (-1,0) -- (0,-1) -- (1,0) -- cycle;
                \draw[latex-] (0.55,0.55) -- (1,1) node[above,xshift=0.7cm] {$x + y = 1$};
                \draw[latex-] (-0.55,0.55) -- (-1,1) node[above,xshift=-0.7cm] {$- x + y = 1$};
                \draw[latex-] (-0.55,-0.55) -- (-1,-1) node[below,xshift=-0.7cm] {$- x - y = 1$};
                \draw[latex-] (0.55,-0.55) -- (1,-1) node[below,xshift=0.7cm] {$x - y = 1$};
            \end{tikzpicture}
        \end{center}
        \item Para graficar la bola, es necesario observar que
        \begin{align*}
            (x, y) \in B(\mathbf{0}, 1) & \Longleftrightarrow \| (x, y) - (0, 0) \|_2 \leq 1 \\
            & \Longleftrightarrow \| (x, y) \|_2 \leq 1 \\
            & \Longleftrightarrow \sqrt{x^2 + y^2} \leq 1
        \end{align*}
        Esta desigualdad describe un círculo de radio $1$ centrado en el origen $(0, 0)$. Por lo tanto, la representación gráfica de esta bola es:
        \begin{center}
            \begin{tikzpicture}
                \draw[-stealth] (-2.5,0) -- (2.5,0) node[below left] {$x$};
                \draw[-stealth] (0,-2.5) -- (0,2.5) node[below left] {$y$};
                \filldraw[cw2,opacity=0.1] (0,0) circle (1cm);
                \draw[cw0] (0,0) circle (1cm);
                \draw[latex-] (0.6,0.9) -- (1.3,1.5) node[above] {$x^2 + y^2 = 1$};
            \end{tikzpicture}
        \end{center}
        \item Para graficar la bola, es necesario observar que
        \begin{align*}
            (x, y) \in B(\mathbf{0}, 1) & \Longleftrightarrow \| (x, y) - (0, 0) \|_{\infty} \leq 1 \\
            & \Longleftrightarrow \| (x, y) \|_{\infty} \leq 1 \\
            & \Longleftrightarrow \max \left\{ |x| + |y| \right\} \leq 1
       \end{align*}
       Esta desigualdad describe un cuadrado centrado en el origen $(0, 0)$, con lados paralelos a los ejes coordenados y longitud de lado $2$. Por lo tanto, la representación gráfica de esta bola es:
       \begin{center}
           \begin{tikzpicture}
               \draw[-stealth] (-2.5,0) -- (2.5,0) node[below left] {$x$};
               \draw[-stealth] (0,-2.5) -- (0,2.5) node[below left] {$y$};
               \filldraw[cw2,opacity=0.1] (-1,-1) rectangle (1,1);
               \draw[cw0] (-1,-1) rectangle (1,1);
               \draw[latex-] (0.5,1.05) -- (0.75,1.75) node[right] {$y = 1$};
               \draw[latex-] (-0.5,-1.05) -- (-0.75,-1.75) node[left] {$y = - 1$};
               \draw[latex-] (1.05,-0.5) -- (1.75,-0.75) node[below] {$x = 1$};
               \draw[latex-] (-1.05,0.5) -- (-1.75,0.75) node[above] {$x = - 1$};
            \end{tikzpicture}
       \end{center}
    \end{enumerate}
    \item Sea $\mathbf{x} \in \RR[n]$ y consideremos la función $\| \phantom{x} \|$. Queremos probar que $\| \phantom{x} \|$ es continua en cualquier punto $\mathbf{x}$, es decir, que si $\mathbf{y} \to \mathbf{x}$, entonces $\| \mathbf{y} \| \to \| \mathbf{x} \|$. Dado $\varepsilon > 0$, tomemos $\delta = \varepsilon$. Si $\| \mathbf{x} - \mathbf{y} \| < \delta$, entonces, por la desigualdad del corolario \ref{corollary:desigualdad_vabsoluto},
    $$\big| \| \mathbf{x} \| - \| \mathbf{y} \| \big| \leq \| \mathbf{x} - \mathbf{y} \| < \delta = \varepsilon.$$
    Esto implica que $\| \mathbf{y} \| \to \| \mathbf{x} \|$ cuando $\mathbf{y} \to \mathbf{x}$, lo que prueba que $\| \phantom{x} \|$ es continua en cualquier punto $\mathbf{x}$.
    \item Sabemos que la sucesión $(\mathbf{x}_n)$ converge a $\mathbf{x}$, es decir,
    $$\lim_{n \to \infty} \mathbf{x}_n = \mathbf{x}.$$
    Por definición de convergencia, para todo $\varepsilon > 0$, existe $N \in \NN$ tal que
    $$\| \mathbf{x}_n - \mathbf{x} \|, \quad \text{ para todo } n \geq N.$$
    En cualquier espacio normado se cumple que
    $$\big| \| \mathbf{x}_n \| - \| \mathbf{x} \| \big| \leq \| \mathbf{x}_n - \mathbf{x} \|.$$
    Dado $\varepsilon > 0$, para todo $n \geq N$,
    $$\big| \| \mathbf{x}_n \| - \| \mathbf{x} \| \big| \leq \| \mathbf{x}_n - \mathbf{x} \| < \varepsilon.$$
    Como $\|\mathbf{x}_n - \mathbf{x}\| \to 0$ por la convergencia de $\mathbf{x}_n$ a $\mathbf{x}$, se sigue que $\big| \|\mathbf{x}_n\| - \|\mathbf{x}\| \big| \to 0$, lo que demuestra que $\|\mathbf{x}_n\| \to \|\mathbf{x}\|$, es decir,
    $$\lim_{n \to \infty} \|\mathbf{x}_n\| = \|\mathbf{x}\|.$$
    \item Supongamos que $\mathbf{x}_n \to \mathbf{x}_0$ en $\RR[k]$. Por definición, para todo $\varepsilon > 0$, existe $N \in \NN$ tal que para $n \geq N$,
    $$\|\mathbf{x}_n - \mathbf{x}_0\| = \sqrt{\sum_{r=1}^k \left(x_n^r - x_0^r\right)^2} < \varepsilon.$$
    Para cada componente $r \in \{1, 2, \dots, k\}$, se cumple
    $$\left|x_n^r - x_0^r\right| \leq \sqrt{\sum_{r=1}^k \left(x_n^r - x_0^r\right)^2} = \|\mathbf{x}_n - \mathbf{x}_0\| < \varepsilon.$$
    Por lo tanto, para todo $r$, $x_n^r \to x_0^r$ en $\RR$. Recíprocamente, supongamos que para cada $r \in \{1, 2, \dots, k\}$, $x_n^r \to x_0^r$ en $\RR$. Dado $\varepsilon > 0$, para cada componente $r$, existe $N_r \in \NN$ tal que para $n \geq N_r$,
    $$\left|x_n^r - x_0^r\right| < \frac{\varepsilon}{\sqrt{k}}.$$
    Sea $N = \max \{N_1, N_2, \dots, N_k\}$. Para $n \geq N$,
    \begin{align*}
        \|\mathbf{x}_n - \mathbf{x}_0\|^2 & = \sum_{r=1}^k \left|x_n^r - x_0^r\right|^2 \\
        & < \sum_{r=1}^k \left(\frac{\varepsilon}{\sqrt{k}}\right)^2 \\
        & = k \cdot \frac{\varepsilon^2}{k} \\
        & = \varepsilon^2
    \end{align*}
    Tomando raíz cuadrada
    $$\|\mathbf{x}_n - \mathbf{x}_0\| < \varepsilon.$$
    Por lo tanto, $\mathbf{x}_n \to \mathbf{x}_0$ en $\RR[k]$.
    \item Supongamos que $(\mathbf{x}_n)$ converge a algún punto $\mathbf{x} \in \RR[k]$. Esto significa que, dado $\varepsilon > 0$, existe $N \in \NN$ tal que para todo $n \geq N$,
    $$\|\mathbf{x}_n - \mathbf{x}\| < \frac{\varepsilon}{2}.$$
    De manera similar, para todo $m \geq N$,
    $$\|\mathbf{x}_m - \mathbf{x}\| < \frac{\varepsilon}{2}.$$
    Aplicando la desigualdad triangular,
    \begin{align*}
        \|\mathbf{x}_n - \mathbf{x}_m\| & = \|\mathbf{x}_n - \mathbf{x} - \mathbf{x}_m + \mathbf{x}\| \\
        & \leq \|\mathbf{x}_n - \mathbf{x}\| + \|\mathbf{x}_m - \mathbf{x}\| \\
        & < \frac{\varepsilon}{2} + \frac{\varepsilon}{2} = \varepsilon
    \end{align*}
    Esto muestra que $(\mathbf{x}_n)$ es una sucesión de Cauchy, pues para todo $\varepsilon > 0$ podemos encontrar un $N$ tal que para $m$, $n \geq N$, se cumple $\|\mathbf{x}_n - \mathbf{x}_m\| < \varepsilon$.
    \item Dada una sucesión de Cauchy $(\mathbf{x}_n)$ en $\RR[k]$, escribimos cada término como
    $$\mathbf{x}_n = \left(x_n^1, x_n^2, \dots, x_n^k\right) \in \RR[k].$$
    Dado que $\|\mathbf{x}_n - \mathbf{x}_m\|$, es decir,
    $$\|\mathbf{x}_n - \mathbf{x}_m\| = \sqrt{\sum_{r = 1}^{k} \left(x_n^r - x_m^r\right)^2}.$$
    La condición de Cauchy implica que cada coordenada $x_n^i$ es una sucesión de Cauchy en $\RR$. Es decir,
    $$\forall \varepsilon > 0, \exists N \text{ tal que } \forall m, n \geq N, \quad \left|x_n^i - x_m^i\right| < \frac{\varepsilon}{\sqrt{k}}.$$
    Pero como $\RR$ es completo, cada sucesión $(x_n^i)$ converge a algún número real $x^i$. Es decir, existe un vector
    $$\mathbf{x} = \left(x^1, x^2, \dots, x^k\right) \in \RR[k]$$
    tal que para cada $i = 1, 2, \dots, k$, $x_n^i \to x^i$ cuando $n \to \infty$. Ahora, probamos que $\mathbf{x}_n \to \mathbf{x}$. Para cualquier $\varepsilon > 0$, dado que cada coordenada converge, existe un $N$ tal que para todo $n \geq N$,
    $$\left|x_n^i - x^i\right| < \frac{\varepsilon}{\sqrt{k}}, \quad \text{ para } i = 1, 2,  \dots, k.$$
    Entonces,
    \begin{align*}
        \|\mathbf{x}_n - \mathbf{x}\| & = \sum_{r = 1}^{k} \left|x_n^r - x^r\right|^2 \\
        & < \sum_{r = 1}^{k} \left( \frac{\varepsilon}{\sqrt{k}} \right)^2 \\
        & = k \cdot \frac{\varepsilon^2}{k} \\
        & = \varepsilon^2
    \end{align*}
    Tomando raíz cuadrada
    $$\|\mathbf{x}_n - \mathbf{x}\| < \varepsilon.$$
    Esto prueba que $\mathbf{x}_n \to \mathbf{x}$, por consiguiente, la sucesión de Cauchy converge en $\RR[k]$. Por lo tanto, hemos demostrado que toda sucesión de Cauchy en $\RR[k]$ converge en $\RR[k]$, lo que significa que $\RR[k]$ es un espacio normado completo. Esto nos permite concluir que $\RR[k]$ es un espacio de Banach.
    \item Sea $\mathbf{x} = (x_1, x_2, \dots, x_n) \in \RR[n]$.
    \begin{enumerate}
        \item Como cada sumando es no negativo, en particular se cumple que
        $$x_i^2 \leq x_1^2 + x_2^2 + \dots + x_n^2.$$
        Tomando raíz cuadrada en ambos lados
        $$|x_i| = \sqrt{x_i^2} \leq \sqrt{x_1^2 + x_2^2 + \dots + x_n^2} = \| \mathbf{x} \|.$$
        Ahora bien, sea $M = \sup \{|x_1|, |x_2|, \dots, |x_n|\}$. Como $|x_i| \leq M$ para todo $i$, se sigue que
        $$x_j^2 \leq M^2.$$
        Sumando sobre todas las componentes,
        $$x_1^2 + x_2^2 + \cdots + x_n^2 \leq n \cdot M^2.$$
        Tomando raíz cuadrada,
        \begin{align*}
            \| \mathbf{x} \| & = \sqrt{x_1^2 + x_2^2 + \cdots + x_n^2} \\
            & \leq \sqrt{n \cdot M^2} \\
            & = \sqrt{n} \cdot M
        \end{align*}
        De esta forma,
        $$|x_i| \leq \| \mathbf{x} \| \leq \sqrt{n} \sup \{ |x_1|, |x_2|, \dots, |x_n|\}.$$
        \item Es el mismo ejercicio que 1b).
    \end{enumerate}
    \item Sea $\mathbf{x} = (x_1, x_2, \dots, x_n) \in \RR[n]$.
    \begin{enumerate}
        \item Es el mismo ejercicio que 1b).
        \item Recordemos que
        $$\|\mathbf{x}\|_1 = |x_1| + |x_2| + \dots + |x_n|.$$
        Como la norma infinito es el máximo de estos valores, tenemos que
        $$\|\mathbf{x}\|_{\infty} \leq \|\mathbf{x}\|_1.$$
        Dado que $|\mathbf{x}|_{\infty} = \max {|x_1|, |x_2|, \dots, |x_n|}$, entonces para todo $i$,
        $$|x_i| \leq \|\mathbf{x}\|_{\infty}.$$
        Sumando sobre todas las componentes,
        $$|x_1| + |x_2| + \dots + |x_n| \leq n \|\mathbf{x}\|_{\infty}.$$
        De esta forma,
        $$\| \mathbf{x} \|_{\infty} \leq \| \mathbf{x} \|_1 \leq n \| \mathbf{x} \|_{\infty}.$$
        \item A partir de los resultados anteriores, tenemos las desigualdades:
        $$\|\mathbf{x}\|_{\infty} \leq \|\mathbf{x}\| \leq \sqrt{n} \|\mathbf{x}\|_{\infty}$$
        y
        $$\|\mathbf{x}\|_{\infty} \leq \|\mathbf{x}\|_1 \leq n \|\mathbf{x}\|_{\infty}.$$
        Para acotar $\|\mathbf{x}\|_1$ en términos de $\|\mathbf{x}\|$, combinemos las desigualdades. Usando $\| \mathbf{x} \| \leq \sqrt{n} \| \mathbf{x} \|_\infty$  y $\| \mathbf{x} \|_\infty \leq \| \mathbf{x} \|_1$,
        $$\| \mathbf{x} \| \leq \sqrt{n} \| \mathbf{x} \|_\infty \leq \sqrt{n} \| \mathbf{x} \|_1.$$
        Usando $\| \mathbf{x} \|_1 \leq n \| \mathbf{x} \|_\infty$ y $\| \mathbf{x} \|_\infty \leq \| \mathbf{x} \|$,
        $$\| \mathbf{x} \|_1 \leq n \| \mathbf{x} \|_\infty \leq n \| \mathbf{x} \| \Longrightarrow \frac{1}{n} \| \mathbf{x} \|_1 \leq \| \mathbf{x} \|.$$
        Uniendo ambas desigualdades,
        $$\frac{1}{n} \| \mathbf{x} \|_1 \leq \| \mathbf{x} \| \leq \sqrt{n} \| \mathbf{x} \|_1, \quad \forall \mathbf{x} \in \RR[n].$$
        Esto muestra que $\|\mathbf{x}\|_1$ y $\|\mathbf{x}\|$ son normas equivalentes.
    \end{enumerate}
    \item \begin{enumerate}
        \item Sea $M = \|\mathbf{x}\|_\infty$. Como $|x_i| \leq M$ para toda $i$, se cumple
        $$\|\mathbf{x}\|_p^p = \sum_{i=1}^n |x_i|^p \geq |x_k|^p = M^p.$$
        Por lo tanto,
        $$\|\mathbf{x}\|_{\infty}^p \leq \|\mathbf{x}\|_p^p.$$
        \item Sea $M = \|\mathbf{x}\|_{\infty}$. Como $|x_i|^p \leq M^p$ para toda $i$, sumando sobre todas las componentes,
        $$\sum_{i=1}^n |x_i|^p \leq n \cdot M^p.$$
        Tomando la raíz $p$-ésima en ambos lados,
        $$\| \mathbf{x} \|_p \leq n^{1/p} \| \mathbf{x} \|_{\infty}.$$
        \item De la desigualdad en a), tomando la raíz $p$-ésima,
        $$\|\mathbf{x}\|_{\infty} \leq \|\mathbf{x}\|_p.$$
        Así,
        $$\|\mathbf{x}\|_{\infty} \leq \|\mathbf{x}\|_p \leq n^{1/p} \|\mathbf{x}\|_{\infty}.$$
        Esto muestra que $\|\mathbf{x}\|_{\infty}$ y $\|\mathbf{x}\|_p$ son normas equivalentes.
    \end{enumerate}
    \item Sea $\mathbf{x} = (-1, -2, 5, -9) \in \RR[4]$.
    \begin{enumerate}
        \item[i)] 9a) Sean $\| \mathbf{x} \|_{\infty} = 9$ y $\| \mathbf{x} \| = \sqrt{111} \approx 10.535$. Entonces,
        $$9 \leq \underbrace{\sqrt{111}}_{10.535} \leq \sqrt{4} \cdot \sqrt{111} \approx 21.071.$$
        9b) Sean $\| \mathbf{x} \|_1 = 17$ y $\| \mathbf{x} \|_{\infty} = 9$. Entonces,
        $$9 \leq 17 \leq 4 \cdot 17 = 68.$$
        9c) Sean $\| \mathbf{x} \| = \sqrt{111} \approx 10.535$ y $\| \mathbf{x} \|_1 = 17$. Entonces,
        $$4.25 = \frac{1}{4} \cdot 17 \leq \underbrace{\sqrt{111}}_{10.535} \leq \sqrt{4} \cdot 17 = 34.$$
        10a) Sea $p = 3$. Sean $\| \mathbf{x} \|_{\infty} = 9$ y $\| \mathbf{x} \|_p = 863^{1/3}$. Entonces,
        $$729 = 9^3 \leq \left(863^{1/3}\right)^3 = 863.$$
        10b) Sea $p = 3$. Sean $\| \mathbf{x} \|_{\infty} = 9$ y $\| \mathbf{x} \|_p = 863^{1/3}$. Entonces,
        $$863 = \left(863^{1/3}\right)^3 \leq 4^{1/3} \cdot 9^3 \approx 1157.215.$$
        10c) Sea $p = 3$. Sean $\| \mathbf{x} \|_{\infty} = 9$ y $\| \mathbf{x} \|_p = 863^{1/3}$. Entonces,
        $$9 \leq \underbrace{863^{1/3}}_{9.520} \leq 4^{1/3} \cdot 9 = 14.286.$$
    \end{enumerate}
    Sea $\mathbf{x} = (15, -2, 4) \in \RR[3]$.
    \begin{enumerate}
        \item[ii)] 9a) Sean $\| \mathbf{x} \|_{\infty} = 15$ y $\| \mathbf{x} \| = 7\sqrt{5} \approx 15.652$. Entonces,
        $$15 \leq \underbrace{7\sqrt{5}}_{15.652} \leq \sqrt{3} \cdot 15 \approx 25.980.$$
        9b) Sean $\| \mathbf{x} \|_1 = 21$ y $\| \mathbf{x} \|_{\infty} = 15$. Entonces,
        $$15 \leq 21 \leq 3 \cdot 15 = 45.$$
        9c) Sean $\| \mathbf{x} \| = 7\sqrt{5} \approx 15.652$ y $\| \mathbf{x} \|_1 = 21$. Entonces,
        $$7 = \frac{1}{3} \cdot 21 \leq \underbrace{7\sqrt{5}}_{15.652} \leq \sqrt{3} \cdot 21 \approx 36.373.$$
        10a) Consideremos $p = 3$. Sean $\| \mathbf{x} \|_{\infty} = 15$ y $\| \mathbf{x} \|_p = 3447^{1/3}$. Entonces,
        $$3375 = 15^3 \leq \left(3447^{1/3}\right)^3 = 3447.$$
        10b) Consideremos $p = 3$. Sean $\| \mathbf{x} \|_{\infty} = 15$ y $\| \mathbf{x} \|_p = 3447^{1/3}$. Entonces,
        $$3447 = \left(3447^{1/3}\right)^3 \leq 3^{1/3} \cdot 15^3 \approx 4867.592.$$
        10c) Consideremos $p = 3$. Sean $\| \mathbf{x} \|_{\infty} = 15$ y $\| \mathbf{x} \|_p = 3447^{1/3}$. Entonces,
        $$15 \leq \underbrace{3447^{1/3}}_{15.105} \leq 3^{1/3} \cdot 15 = 21.633.$$
    \end{enumerate}
\end{enumerate}

%\end{multicols}