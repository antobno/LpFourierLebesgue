\chapter{ESPACIOS MÉTRICOS}
\printchaptertableofcontents

Al notar que los espacios vectoriales requieren más estructura para abordar conceptos topológicos, los espacios métricos surgen como una herramienta clave al introducir una medida de distancia. Esto permite explorar nociones como apertura, clausura, convergencia y completitud en un marco más general, facilitando un análisis más profundo de sus propiedades geométricas y analíticas. Los espacios métricos, al generalizar la idea intuitiva de distancia que conocemos en contextos como la recta real o el plano euclidiano, nos ofrecen un entorno versátil para estudiar fenómenos matemáticos que trascienden las limitaciones de las estructuras puramente algebraicas. En este capítulo, estudiaremos los espacios métricos como un pilar fundamental que establece conexiones profundas con otras áreas de las matemáticas, como la topología y el análisis funcional.

Comenzaremos definiendo los espacios métricos y examinando propiedades básicas como las bolas abiertas, que servirán como una herramienta clave para entender la topología en este contexto. Seguiremos con el estudio de los conjuntos abiertos y cerrados, fundamentales para comprender la estructura de estos espacios. Posteriormente, abordaremos la continuidad, un concepto que extiende las ideas del cálculo a espacios más generales, y presentaremos los espacios topológicos, que nos introducirán a ideas más avanzadas.

Continuaremos explorando la convergencia y la completitud, analizando cómo las sucesiones se comportan y qué implica que un espacio sea completo. Luego, investigaremos la completación de espacios métricos, un proceso que permite construir espacios completos a partir de otros, y estudiaremos la compacidad, una propiedad que caracteriza espacios con un comportamiento de “finitud” topológica. También abordaremos la compacidad sucesional, que ofrece una perspectiva adicional sobre esta noción. Seguidamente, nos detendremos en los teoremas de Heine-Borel y Arzelà-Ascoli, resultados clásicos con aplicaciones importantes en análisis, y concluiremos con la conexidad, un concepto que nos ayuda a entender la continuidad estructural de los espacios.

\newpage

\section{Espacios vectoriales normados}

Los espacios vectoriales de interés en el análisis abstracto y en aplicaciones tienen una estructura mucho más rica que la descrita únicamente por los ocho axiomas principales. Los axiomas de espacio vectorial solo describen las propiedades algebraicas de los elementos del espacio: suma, multiplicación por escalares y combinaciones de estas. Sin embargo, faltan los conceptos topológicos como apertura, clausura, convergencia y completitud. Estos conceptos pueden introducirse mediante una medida de distancia en el espacio.

\begin{definicion}{}{normaV}
    Sea $V$ un espacio vectorial sobre $K$. Se define la \emph{norma} en $V$ como una función $\| \phantom{x} \| : V \longrightarrow \RR$ que satisface las siguientes propiedades:
    \begin{enumerate}[label=\roman*), topsep=6pt, itemsep=0pt]
        \item $\| \mathbf{x} \| \geq 0$, para todo $\mathbf{x} \in V$.
        \item $\| \mathbf{x} \| = 0$ si y solo si $\mathbf{x} = \mathbf{0}$.
        \item $\| \alpha \mathbf{x} \| = |\alpha| \| \mathbf{x} \|$, para todo escalar $\alpha$ y $\mathbf{x} \in V$.
        \item $\| \mathbf{x} + \mathbf{y} \| \leq \| \mathbf{x} \| + \| \mathbf{y} \|$, para todo $\mathbf{x}$, $\mathbf{y} \in V$. \hfill (desigualdad del triángulo)
    \end{enumerate}
\end{definicion}

\begin{examplebox}{}{}
    En $\RR$, sea $\| \mathbf{x} \| = |x|$, probemos entonces que $\| \phantom{x} \|$ cumple las cuatro propiedades establecidas en la definición \ref{definicion:normaV}.
    \begin{enumerate}[label=\roman*), topsep=6pt, itemsep=0pt]
        \item Procedamos por casos.
        \begin{enumerate}
            \item Si $x \geq 0$, se define $|x| = x$. Como $x \geq 0$, se tiene $|x| \geq 0$.
            \item Si $x < 0$, se define $|x| = -x$. Como $x$ es negativo, $-x$ es positivo, es decir, $|x| \geq 0$.
        \end{enumerate}
        En ambos casos se concluye que $|x| \geq 0$ para todo $x \in \RR$.
        \item Del inciso anterior, $|x| = 0$ no puede ocurrir a menos que $x = 0$.
        \item Consideremos dos casos respecto a $\alpha$.
        \begin{enumerate}
            \item Sea $\alpha \geq 0$, entonces $|\alpha| = \alpha$. Si $x \geq 0$, entonces $\alpha x \geq 0$ y
            $$|\alpha x| = \alpha x = \alpha |x|.$$
            Si $x < 0$, entonces $\alpha x \leq 0$ y
            $$|\alpha x| = -(\alpha x) = \alpha (-x) = \alpha |x|.$$
            \item Sea $\alpha < 0$, entonces $|\alpha| = -\alpha$. Si $x \geq 0$, entonces $\alpha x \leq 0$ y
            $$|\alpha x| = -(\alpha x) = (-\alpha) x = |\alpha| |x|.$$
            Si $x < 0$, entonces $\alpha x \geq 0$ y
            $$|\alpha x| = \alpha x = (-\alpha)(-x) = |\alpha| |x|.$$
        \end{enumerate}
        En todos los casos se obtiene que $|\alpha x| = |\alpha| |x|$.
        \item Consideremos que
        $$(|x| + |y|)^2 = x^2 + 2|x||y| + y^2,$$
        y
        $$|x + y|^2 = (x + y)^2 = x^2 + 2xy + y^2.$$
        Dado que $2xy \leq 2|x||y|$, se tiene $|x + y|^2 \leq (|x| + |y|)^2$. Al tomar la raíz cuadrada de ambos lados se concluye que
        $$|x + y| \leq |x| + |y|.$$
    \end{enumerate}
    Dado que el valor absoluto en $\RR$ cumple las cuatro propiedades anteriores, concluimos que $|\phantom{x}|$ es una norma en $\RR$.
\end{examplebox}

\newpage

Hemos demostrado que el valor absoluto en $\RR$ cumple con las propiedades fundamentales de una norma. A partir de esta propiedad, se puede extender la idea de norma a espacios de mayor dimensión, como $\RR[n]$, y demostrar que ciertas definiciones, como la norma $\| \phantom{x} \|_1$, también cumplen con los requisitos necesarios para ser consideradas normas en este contexto.

\begin{examplebox}{}{}
    Sea $\| \phantom{x} \|_1$ la función definida en $\RR[n]$ por
    $$\| \mathbf{x} \|_1 = |x_1| + |x_2| + \cdots + |x_n|,$$
    para cada $\mathbf{x} = (x_1, x_2, \dots, x_n)$. Utilizando las propiedades del valor absoluto como base, demostremos que $\| \phantom{x} \|_1$ cumple las cuatro propiedades establecidas en la definición \ref{definicion:normaV}.
    \begin{enumerate}[label=\roman*), topsep=6pt, itemsep=0pt]
        \item Como el valor absoluto es siempre no negativo, se tiene que $|x_i| \geq 0$ para cada $i \in \{ 1, 2, \dots, n \}$. Al sumar estos valores, se obtiene
        $$\| \mathbf{x} \|_1 = |x_1| + |x_2| + \cdots + |x_n| \geq 0.$$
        \item Si $\| \mathbf{x} \|_1 = 0$, entonces
        $$|x_1| + |x_2| + \cdots + |x_n| = 0.$$
        Como cada término $|x_i| \geq 0$, esto solo es posible si $|x_i| = 0$, es decir, $x_i = 0$ para cada $i \in \{ 1, 2, \dots, n \}$. Por lo tanto, $\mathbf{x} = \mathbf{0}$. Recíprocamente, si $\mathbf{x} = \mathbf{0}$, entonces $x_i = 0$ para cada $i \in \{ 1, 2, \dots, n \}$, lo que implica que $\| \mathbf{x} \|_1 = 0$.
        \item Sea $\mathbf{x} = (x_1, x_2, \dots, x_n)$, entonces $\alpha \mathbf{x} = (\alpha x_1, \alpha x_2, \dots, \alpha x_n)$. Por la propiedad del valor absoluto, se tiene que $|\alpha x_i| = |\alpha| |x_i|$, por lo que
        \begin{align*}
            \| \alpha \mathbf{x} \|_1 & = |\alpha x_1| + |\alpha x_2| + \cdots + |\alpha x_n| \\
            & = |\alpha| |x_1| + |\alpha| |x_2| + \cdots + |\alpha| |x_n| \\
            & = |\alpha| (|x_1| + |x_2| + \cdots + |x_n|) \\
            & = |\alpha| \| \mathbf{x} \|_1
        \end{align*}
        \item Tenemos que $\mathbf{x} + \mathbf{y} = (x_1 + y_1, x_2 + y_2, \dots, x_n + y_n)$. Al usar la desigualdad triangular para cada $i \in \{ 1, 2, \dots, n \}$ en $\RR$,
        $$|x_i + y_i| \leq |x_i| + |y_i|$$
        obtenemos
        \begin{align*}
            \| \mathbf{x} + \mathbf{y} \|_1 & = |x_1 + y_1| + |x_2 + y_2| + \cdots + |x_n + y_n| \\
            & \leq |x_1| + |y_1| + |x_2| + |y_2| + \cdots + |x_n| + |y_n| \\
            & = |x_1| + |x_2| + \cdots + |x_n| + |y_1| + |y_2| + \cdots + |y_n| \\
            & = \| \mathbf{x} \|_1 + \| \mathbf{y} \|_1
        \end{align*}
    \end{enumerate}
    Dado que $\| \phantom{x} \|_1$ en $\RR[n]$ cumple las cuatro propiedades anteriores, concluimos que $\| \phantom{x} \|_1$ es una norma en $\RR[n]$.
\end{examplebox}

Utilizando las propiedades de los números reales como base, procederemos a mostrar que la norma $\| \phantom{x} \|_2$ en $\RR[n]$, definida como la raíz cuadrada de la suma de los cuadrados de las componentes de un vector, también satisface las cuatro condiciones esenciales que definen a una norma en un espacio vectorial.

\begin{examplebox}{}{}
    Sea $\| \phantom{x} \|_2$ la función definida en $\RR[n]$ por
    $$\| \mathbf{x} \|_2 = \sqrt{x_1^2 + x_2^2 + \cdots + x_n^2},$$
    para cada $\mathbf{x} = (x_1, x_2, \dots, x_n)$. Demostremos que $\| \phantom{x} \|_2$ cumple las cuatro propiedades establecidas en la definición \ref{definicion:normaV}.
    \begin{enumerate}[label=\roman*), topsep=6pt, itemsep=0pt]
        \item Dado que para cada $i \in \{ 1, 2, \dots, n \}$, $x_i^2 \geq 0$, se sigue que
        $$x_1^2 + x_2^2 + \dots + x_n^2 \geq 0.$$
        Esto nos lleva a que
        $$\| \mathbf{x} \|_2 = \sqrt{x_1^2 + x_2^2 + \dots + x_n^2} \geq 0.$$
        \item Del inciso anterior, si $\| \mathbf{x} \|_2 = 0$, entonces
        $$x_1^2 + x_2^2 + \dots + x_n^2 = 0.$$
        Esto ocurre si y solo si $x_i = 0$ para cada $i \in \{ 1, 2, \dots, n \}$, es decir, $\mathbf{x} = \mathbf{0}$.
        \item Sabiendo que para cualquier escalar, se cumple $\sqrt{\alpha^2} = |\alpha|$, tenemos que
        \begin{align*}
            \|\alpha \mathbf{x}\|_2 & = \sqrt{(\alpha x_1)^2 + (\alpha x_2)^2 + \dots + (\alpha x_n)^2} \\
            & = \sqrt{\alpha^2 x_1^2 + \alpha^2 x_2^2 + \dots + \alpha^2 x_n^2} \\
            & = \sqrt{\alpha^2 \left(x_1^2 + x_2^2 + \dots + x_n^2\right)} \\
            & = |\alpha| \sqrt{x_1^2 + x_2^2 + \dots + x_n^2} \\
            & = |\alpha| \| \mathbf{x} \|_2
        \end{align*}
        \item Tenemos que $\mathbf{x} + \mathbf{y} = (x_1 + y_1, x_2 + y_2, \dots, x_n + y_n)$ y por definición,
        $$\| \mathbf{x} + \mathbf{y} \|_2 = \sqrt{(x_1 + y_1)^2 + (x_2 + y_2)^2 + \cdots + (x_n + y_n)^2}.$$
        Usamos la desigualdad básica
        $$(x_i + y_i)^2 \leq 2\left(x_i^2 + y_i^2\right),$$
        la cual se obtiene de
        $$0 \leq (x_i - y_i)^2 = x_i^2 - 2x_iy_i + y_i^2.$$
        Reordenando y sumando para todas las componentes,
        $$2 \sum_{i=1}^{n} x_i y_i \leq \sum_{i=1}^{n} x_i^2 + \sum_{i=1}^{n} y_i^2.$$
        Lo que conlleva a que
        \begin{align*}
            \| \mathbf{x} + \mathbf{y} \|_2 & = \sqrt{\sum_{i=1}^{n} x_i^2 + 2 \sum_{i=1}^{n} x_i y_i + \sum_{i=1}^{n} y_i^2} \\
            & \leq \sqrt{2\sum_{i=1}^{n} x_i^2 + 2\sum_{i=1}^{n} y_i^2} \\
            & = \sqrt{2 \left( \sum_{i=1}^{n} x_i^2 + \sum_{i=1}^{n} y_i^2 \right)} \\
            & \leq \sqrt{\sum_{i=1}^{n} x_i^2 + \sum_{i=1}^{n} y_i^2} \\
            & \leq \sqrt{x_1^2 + x_2^2 + \cdots + x_n^2} + \sqrt{y_1^2 + y_2^2 + \cdots + y_n^2} \\
            & = \| \mathbf{x} \|_1 + \| \mathbf{y} \|_1
        \end{align*}
    \end{enumerate}
    Dado que $\| \phantom{x} \|_2$ en $\RR[n]$ cumple las cuatro propiedades anteriores, concluimos que $\| \phantom{x} \|_2$ es una norma en $\RR[n]$.
\end{examplebox}

\newpage

La norma infinita se define como el valor máximo de los valores absolutos de las componentes del vector y cumple con las cuatro propiedades esenciales de una norma en un espacio vectorial.

\begin{examplebox}{}{}
    Sea $\| \phantom{x} \|_{\infty}$ la función definida en $\RR[n]$ por
    $$\| \mathbf{x} \|_{\infty} = \max \{ |x_1|, |x_2|, \dots, |x_n| \},$$
    para cada $\mathbf{x} = (x_1, x_2, \dots, x_n)$. Demostremos que $\| \phantom{x} \|_{\infty}$ cumple las cuatro propiedades establecidas en la definición \ref{definicion:normaV}.
    \begin{enumerate}[label=\roman*), topsep=6pt, itemsep=0pt]
        \item Como el valor absoluto es siempre no negativo, se tiene que $|x_i| \geq 0$ para cada $i \in \{ 1, 2, \dots, n \}$. De lo anterior dicho, se concluye que
        $$\| \mathbf{x} \|_{\infty} = \max \{ |x_1|, |x_2|, \dots, |x_n| \} \geq 0.$$
        \item Si $\| \mathbf{x} \|_{\infty} = 0$, entonces
        $$\max \{ |x_1|, |x_2|, \dots, |x_n| \} = 0.$$
        Como el máximo de un conjunto de números no negativos es cero solo si todos los elementos son cero, se deduce que para cada $i \in \{ 1, 2, \dots, n \}$, $|x_i| = 0$, es decir, $x_i = 0$. Por lo tanto, $\mathbf{x} = \mathbf{0}$. Recíprocamente, si $\mathbf{x} = \mathbf{0}$, entonces $x_i = 0$ para cada $i \in \{ 1, 2, \dots, n \}$, lo que implica que $\| \mathbf{x} \|_\infty = 0$.
        \item Sea $\mathbf{x} = (x_1, x_2, \dots, x_n)$, entonces $\alpha \mathbf{x} = (\alpha x_1, \alpha x_2, \dots, \alpha x_n)$. Por la propiedad del valor absoluto, se tiene que $|\alpha x_i| = |\alpha| |x_i|$, por lo que
        \begin{align*}
            \| \alpha \mathbf{x} \|_{\infty} & = \max \{ |\alpha x_1|, |\alpha x_2|, \dots, |\alpha x_n| \} \\
            & = \max \{ |\alpha| |x_1|, |\alpha| |x_2|, \dots, |\alpha| |x_n| \} \\
            & = |\alpha| \max \{ |x_1|, |x_2|, \dots, |x_n| \} \\
            & = |\alpha| \| \mathbf{x} \|
        \end{align*}
        \item Se deja como ejercicio al lector.
    \end{enumerate}
    Dado que $\| \phantom{x} \|_{\infty}$ en $\RR[n]$ cumple las cuatro propiedades anteriores, concluimos que $\| \phantom{x} \|_{\infty}$ es una norma en $\RR[n]$.
\end{examplebox}

Ahora veremos una norma en un espacio vectorial diferente.

\begin{examplebox}{}{}
    Sea $\| \phantom{x} \|$ la función definida en $C[a, b]$ por
    $$\| \mathbf{f} \| = \max_{t \in [a, b]} |f(t)|,$$
    para cada $\mathbf{f} = f(t)$. Utilizando las propiedades del valor absoluto, demostremos que $\| \phantom{x} \|$ cumple las cuatro propiedades establecidas en la definición \ref{definicion:normaV}.
    \begin{enumerate}[label=\roman*), topsep=6pt, itemsep=0pt]
        \item Dado que el valor absoluto es siempre no negativo, $\| \mathbf{f} \| \geq 0$ para todo $\mathbf{f}$.
        \item Del inciso anterior, $\| \mathbf{f} \| = 0$ si y solo si $|f(t)| = 0$ para todo $t \in [a, b]$, lo cual ocurre solo cuando $f(t) = 0$ para todos los $t \in [a, b]$, es decir, $\mathbf{f} = \mathbf{0}$.
        \item Si $f \in C[a, b]$ y $\alpha \in \RR$, entonces
        \begin{align*}
            \| \alpha \mathbf{f} \| & = \max_{t \in [a, b]} |\alpha f(t)| \\
            & = |\alpha| \max_{t \in [a, b]} |f(t)| \\
            & = |\alpha| \| \mathbf{f} \|.
        \end{align*}
        \item Se deja como ejercicio al lector.
    \end{enumerate}
    Dado que $\| \phantom{x} \|$ en $C[a, b]$ cumple las cuatro propiedades anteriores, concluimos que $\| \phantom{x} \|$ es una norma en $C[a, b]$.
\end{examplebox}

\newpage

\begin{definicion}{}{}
    Un \emph{espacio normado} es un espacio vectorial $V$ con una norma definida en $V$. Un espacio normado se denota como $(V, \| \phantom{x} \|)$.
\end{definicion}

Ahora, tras la definición de espacio normado, podemos proporcionar ejemplos de tales espacios utilizando las normas previamente discutidas.

\begin{examplebox}{}{}
    A continuación, presentamos algunos espacios normados basados en las normas definidas anteriormente:
    \begin{enumerate}[label=\roman*), topsep=6pt, itemsep=0pt]
        \item El espacio $\RR$ con la norma del valor absoluto: El par $(\RR, |\phantom{x}|)$ es un espacio normado, como se demostró en el primer ejemplo. Aquí, la norma coincide con el valor absoluto clásico.
        \item El espacio $\RR[n]$ con la norma $\| \phantom{x} \|_1$: El espacio $(\RR[n], \| \phantom{x} \|_1)$ es normado, como se verificó en el segundo ejemplo. Esta norma suma las magnitudes de las componentes del vector.
        \item El espacio $\RR[n]$ con la norma euclidiana $\| \phantom{x} \|_2$: El par $(\RR[n], \| \phantom{x} \|_2)$ también es un espacio normado, como se mostró en el tercer ejemplo. Esta norma corresponde a la distancia euclidiana.
        \item El espacio $\RR[n]$ con la norma infinito $\| \phantom{x} \|_\infty$: El espacio $(\RR[n], \| \phantom{x} \|_\infty)$, donde la norma es el máximo valor absoluto de las componentes, es normado, como se ilustró en el cuarto ejemplo.
        \item El espacio $C[a, b]$ con la norma del máximo: El espacio $(C[a, b], \| \phantom{x} \|)$, con $\displaystyle \|\mathbf{f}\| = \max_{t \in [a,b]} |f(t)|$, es normado, como se comprobó en el quinto ejemplo. Esta norma mide la máxima desviación de una función en el intervalo.
    \end{enumerate}
    Cada uno de estos pares $(V, \| \phantom{x} \|)$ constituye un espacio normado, ya que la función $\| \phantom{x} \|$ satisface los cuatro axiomas de la definición. Es importante notar que un mismo espacio vectorial puede tener múltiples normas, dando lugar a diferentes espacios normados (por ejemplo, $\RR[n]$ con $\| \phantom{x} \|_1$, $\| \phantom{x} \|_2$ o $\| \phantom{x} \|_\infty$).
\end{examplebox}

La siguiente desigualdad útil es una consecuencia directa de la desigualdad triangular.

\begin{theorem}{}{}
    Sea $V$ un espacio normado. Entonces se cumple que
    $$\| \mathbf{x} \| - \| \mathbf{y} \| \leq \| \mathbf{x} - \mathbf{y} \|$$
    para cualesquiera $\mathbf{x}$, $\mathbf{y} \in V$.

    \tcblower
    \demostracion Consideremos los vectores $\mathbf{x}$ y $\mathbf{y}$ en el espacio normado $V$. Usando la desigualdad triangular,
    $$\| \mathbf{x} \| = \| \mathbf{x} - \mathbf{y} + \mathbf{y} \| \leq \| \mathbf{x} - \mathbf{y} \| + \| \mathbf{y} \|.$$
    Reorganizando esta desigualdad, obtenemos
    $$\| \mathbf{x} \| - \| \mathbf{y} \| \leq \| \mathbf{x} - \mathbf{y} \|.$$
\end{theorem}

\begin{corollary}{}{desigualdad_vabsoluto}
    En particular, en un espacio normado $V$, se cumple que para cualesquiera $\mathbf{x}$, $\mathbf{y} \in V$,
    $$\big| \| \mathbf{x} \| - \| \mathbf{y} \| \big| \leq \| \mathbf{x} - \mathbf{y} \|.$$

    \tcblower
    \demostracion Solo basta probar que $\| \mathbf{x} - \mathbf{y} \| \leq \| \mathbf{x} \| - \| \mathbf{y} \|$. Así,
    $$\| \mathbf{y} \| = \| \mathbf{y} - \mathbf{x} + \mathbf{x} \| \leq \| \mathbf{y} - \mathbf{x} \| + \| \mathbf{x} \| = \| \mathbf{x} - \mathbf{y} \| + \| \mathbf{x} \|.$$
    Reorganizando esta desigualdad, obtenemos, $\| \mathbf{y} \| - \| \mathbf{x} \| \leq \| \mathbf{x} - \mathbf{y} \|$. Combinando estas dos desigualdades, $\| \mathbf{x} - \mathbf{y} \| \leq \| \mathbf{x} \| - \| \mathbf{y} \| \leq \| \mathbf{x} - \mathbf{y} \|$, lo que equivale a
    $$\big| \| \mathbf{x} \| - \| \mathbf{y} \| \big| \leq \| \mathbf{x} - \mathbf{y} \|.$$
\end{corollary}

\newpage

\section{Espacios métricos y topología básica}

\begin{definicion}{}{metrica}
    Sea $X$ un conjunto. Una función $d: X \times X \longrightarrow \RR$ se llama \emph{distancia} o \emph{métrica} en $X$ si cumple las siguientes propiedades:
    \begin{enumerate}[label=\roman*), topsep=6pt, itemsep=0pt]
        \item $d(\mathbf{x}, \mathbf{y}) \geq 0$, para todo $\mathbf{x}$, $\mathbf{y} \in X$,
        \item $d(\mathbf{x}, \mathbf{y}) = 0$ si y solo si $\mathbf{x} = \mathbf{y}$,
        \item $d(\mathbf{x}, \mathbf{y}) = d(\mathbf{y}, \mathbf{x})$, para todo $\mathbf{x}$, $\mathbf{y} \in X$,
        \item $d(\mathbf{x}, \mathbf{y}) \leq d(\mathbf{x}, \mathbf{z}) + d(\mathbf{z}, \mathbf{y})$, para todo $\mathbf{x}$, $\mathbf{y}$, $\mathbf{z} \in X$.
    \end{enumerate}
\end{definicion}

\begin{examplebox}{}{}
    Sea $V$ un espacio vectorial sobre $K$ con una norma $\| \phantom{x} \|$. Entonces la función $d: V \times V \longrightarrow \RR$ definida mediante la siguiente regla es una métrica en $V$:
    $$d(\mathbf{x}, \mathbf{y}) = \| \mathbf{x} - \mathbf{y} \|.$$

    \tcblower
    \demostracion
    \begin{enumerate}[label=\roman*), topsep=6pt, itemsep=0pt]
        \item Sabemos que la norma de cualquier vector es siempre no negativa, es decir, $\| \mathbf{x} - \mathbf{y} \| \geq 0$ para todo $\mathbf{x}$, $\mathbf{y} \in V$. Por lo tanto, $d(\mathbf{x}, \mathbf{y}) = \| \mathbf{x} - \mathbf{y} \| \geq 0$.
        \item Procedamos por casos.
        \begin{enumerate}
            \item Si $\mathbf{x} = \mathbf{y}$, entonces $\mathbf{x} - \mathbf{y} = \mathbf{0}$, y por la propiedad de la norma, tenemos que $\| \mathbf{x} - \mathbf{y} \| = 0$. Por lo tanto, $d(\mathbf{x}, \mathbf{y}) = 0$.
            \item Si $d(\mathbf{x}, \mathbf{y}) = 0$, entonces $\| \mathbf{x} - \mathbf{y} \| = 0$. Como la norma de un vector es cero si y solo si el vector es el vector cero, esto implica que $\mathbf{x} - \mathbf{y} = \mathbf{0}$, lo cual significa que $\mathbf{x} = \mathbf{y}$.
        \end{enumerate}
        Por lo tanto, $d(\mathbf{x}, \mathbf{y}) = 0$ si y solo si $\mathbf{x} = \mathbf{y}$.
        \item De acuerdo con la definición de $d$,
        \begin{align*}
            d(\mathbf{x}, \mathbf{y}) & = \| \mathbf{x} - \mathbf{y} \| \\
            & = |-1| \| \mathbf{y} - \mathbf{x} \| \\
            & = \| \mathbf{y} - \mathbf{x} \| \\
            & = d(\mathbf{y}, \mathbf{x})
        \end{align*}
        \item Sean $\mathbf{x}$, $\mathbf{y}$, $\mathbf{z} \in V$, y de acuerdo con las propiedades de la norma,
        \begin{align*}
            d(\mathbf{x}, \mathbf{y}) & = \| \mathbf{x} - \mathbf{y} \| \\
            & = \| (\mathbf{x} - \mathbf{z}) + (\mathbf{z} - \mathbf{y}) \| \\
            & \leq \| \mathbf{x} - \mathbf{z} \| + \| \mathbf{z} - \mathbf{y} \| \\
            & = d(\mathbf{x}, \mathbf{z}) + d(\mathbf{z}, \mathbf{y})
        \end{align*}
    \end{enumerate}
    Por lo tanto, $d(\mathbf{x}, \mathbf{y}) = \| \mathbf{x} - \mathbf{y} \|$ satisface todas las propiedades de una métrica.
\end{examplebox}

En general, si $V$ es un espacio vectorial sobre $K$ con una norma $\| \phantom{x} \|$, la distancia o métrica inducida por la norma $\| \phantom{x} \|$ se define mediante la siguiente fórmula:
$$d(\mathbf{x}, \mathbf{y}) = \| \mathbf{x} - \mathbf{y} \|.$$
Esta construcción es muy útil porque nos permite medir distancias en espacios vectoriales de manera natural, aprovechando la estructura de la norma.

\begin{examplebox}{}{}
    En el conjunto de los números reales, $\RR$, la expresión
    $$d(\mathbf{x}, \mathbf{y}) = \| \mathbf{x} - \mathbf{y} \| = |x - y|$$
    constituye una métrica.
\end{examplebox}

\newpage

\begin{examplebox}{}{}
    Sea $\RR[n]$ y sean $\mathbf{x} = (x_1, x_2, \dots, x_n)$, $\mathbf{y} = (y_1, y_2, \dots, y_n) \in \RR[n]$. La expresión
    $$d_1( \mathbf{x}, \mathbf{y} ) = |x_1 - y_1| + |x_2 - y_2| + \cdots + |x_n - y_n|$$
    constituye una métrica. Sean $\mathbf{x} = (x_1, x_2, \dots, x_n)$, $\mathbf{y} = (y_1, y_2, \dots, y_n) \in \RR[n]$,
    $$\mathbf{x} - \mathbf{y} = (x_1 - y_1, x_2 - y_2, \dots, x_n - y_n)$$
    de donde se sigue
    \begin{align*}
        d_1( \mathbf{x}, \mathbf{y} ) & = \| (x_1 - y_1, x_2 - y_2, \dots, x_n - y_n) \|_1 \\
        & = |x_1 - y_1| + |x_2 - y_2| + \cdots + |x_n - y_n|
    \end{align*}
    Por tanto, la distancia o métrica $d_1$ es inducida por la norma $\| \phantom{x} \|_1$.
\end{examplebox}

\begin{examplebox}{}{}
    Sea $\RR[n]$ y sean $\mathbf{x} = (x_1, x_2, \dots, x_n)$, $\mathbf{y} = (y_1, y_2, \dots, y_n) \in \RR[n]$. La expresión
    $$d_2( \mathbf{x}, \mathbf{y} ) = \sqrt{(x_1 - y_1)^2 + (x_2 - y_2)^2 + \cdots + (x_n - y_n)^2}$$
    constituye una métrica. Sean $\mathbf{x} = (x_1, x_2, \dots, x_n)$, $\mathbf{y} = (y_1, y_2, \dots, y_n) \in \RR[n]$,
    $$\mathbf{x} - \mathbf{y} = (x_1 - y_1, x_2 - y_2, \dots, x_n - y_n)$$
    de donde se sigue
    \begin{align*}
        d_2( \mathbf{x}, \mathbf{y} ) & = \| (x_1 - y_1, x_2 - y_2, \dots, x_n - y_n) \|_2 \\
        & = \sqrt{(x_1 - y_1)^2 + (x_2 - y_2)^2 + \cdots + (x_n - y_n)^2}
    \end{align*}
    Por tanto, la distancia o métrica $d_2$ es inducida por la norma $\| \phantom{x} \|_2$.
\end{examplebox}

\begin{examplebox}{}{}
    Sea $\RR[n]$ y sean $\mathbf{x} = (x_1, x_2, \dots, x_n)$, $\mathbf{y} = (y_1, y_2, \dots, y_n) \in \RR[n]$. La expresión
    $$d_{\infty}( \mathbf{x}, \mathbf{y} ) = \max \left\{ |x_1 - y_1|, |x_2 - y_2|, \dots, |x_n - y_n| \right\}$$
    constituye una métrica. Sean $\mathbf{x} = (x_1, x_2, \dots, x_n)$, $\mathbf{y} = (y_1, y_2, \dots, y_n) \in \RR[n]$,
    $$\mathbf{x} - \mathbf{y} = (x_1 - y_1, x_2 - y_2, \dots, x_n - y_n)$$
    de donde se sigue
    \begin{align*}
        d_{\infty}( \mathbf{x}, \mathbf{y} ) & = \| (x_1 - y_1, x_2 - y_2, \dots, x_n - y_n) \|_{\infty} \\
        & = \max \left\{ |x_1 - y_1|, |x_2 - y_2|, \dots, |x_n - y_n| \right\}
    \end{align*}
    Por tanto, la distancia o métrica $d_{\infty}$ es inducida por la norma $\| \phantom{x} \|_{\infty}$.
\end{examplebox}

\begin{examplebox}{}{}
    Probemos que en el conjunto $X$, la expresión
    $$d(\mathbf{x}, \mathbf{y}) = \begin{cases}
        0 & \text{si } \mathbf{x} = \mathbf{y} \\
        1 & \text{si } \mathbf{x} \neq \mathbf{y}
    \end{cases}$$
    constituye una métrica en $X$.

    \tcblower
    \demostracion Probemos que dicha expresión cumple las cuatro propiedades de la definición \ref{definicion:metrica}.
    \begin{enumerate}[label=\roman*), topsep=6pt, itemsep=0pt]
        \item Sean $\mathbf{x}$, $\mathbf{y} \in X$, entonces
        $$d(\mathbf{x}, \mathbf{y}) = 0 \quad \text{si } \mathbf{x} = \mathbf{y},$$
        o
        $$d(\mathbf{x}, \mathbf{y}) = 1 \quad \text{si } \mathbf{x} \neq \mathbf{y}.$$
        Así, en cualquier caso, $d(\mathbf{x}, \mathbf{y}) \geq 0$.
        \item Por definición de la métrica, $d(\mathbf{x}, \mathbf{y}) = 0$ si y solo si $\mathbf{x} = \mathbf{y}$.
        \item Sean $\mathbf{x}$, $\mathbf{y} \in X$. Procedamos por casos:
        \begin{enumerate}
            \item Si $\mathbf{x} = \mathbf{y}$, entonces $d(\mathbf{x}, \mathbf{y}) = 0$, pero $\mathbf{x} = \mathbf{y}$, entonces $\mathbf{y} = \mathbf{x}$. Por tanto, $d(\mathbf{y}, \mathbf{x}) = 0$, por lo que $d(\mathbf{x}, \mathbf{y}) = d(\mathbf{y}, \mathbf{x})$.
            \item Si $\mathbf{x} \neq \mathbf{y}$, entonces $d(\mathbf{x}, \mathbf{y}) = 1$, pero $\mathbf{x} \neq \mathbf{y}$, entonces $\mathbf{y} \neq \mathbf{x}$. Por tanto, $d(\mathbf{y}, \mathbf{x}) = 1$, por lo que $d(\mathbf{x}, \mathbf{y}) = d(\mathbf{y}, \mathbf{x})$.
        \end{enumerate}
        Así, en cualquiera de los casos, $d(\mathbf{x}, \mathbf{y}) = d(\mathbf{y}, \mathbf{x})$.
        \item Sean $\mathbf{x}$, $\mathbf{y} \in X$. Procedamos por casos:
        \begin{enumerate}
            \item Si $\mathbf{x} = \mathbf{y}$, entonces $d(\mathbf{x}, \mathbf{y}) = 0$, por lo que
            \begin{align*}
                0 & = d(\mathbf{x}, \mathbf{y}) \\
                & \leq d(\mathbf{x}, \mathbf{z}) + d(\mathbf{z}, \mathbf{y})
            \end{align*}
            \item Si $\mathbf{x} \neq \mathbf{y}$, entonces $d(\mathbf{x}, \mathbf{y}) = 1$, por lo que
            \begin{align*}
                1 & = d(\mathbf{x}, \mathbf{y}) \\
                & \leq d(\mathbf{x}, \mathbf{z}) + d(\mathbf{z}, \mathbf{y})
            \end{align*}
            Notemos que $d(\mathbf{x}, \mathbf{z}) = 0$ y $d(\mathbf{z}, \mathbf{y}) = 0$ no ocurre, pues si $d(\mathbf{x}, \mathbf{z}) = 0$ y $d(\mathbf{z}, \mathbf{y}) = 0$, entonces $\mathbf{x} = \mathbf{z}$ y $\mathbf{z} = \mathbf{y}$. Por lo tanto, $\mathbf{x} = \mathbf{y}$ lo cual no puede ser.
        \end{enumerate}
        Así, en cualquiera de los casos, $d(\mathbf{x}, \mathbf{y}) \leq d(\mathbf{x}, \mathbf{z}) + d(\mathbf{z}, \mathbf{y})$.
    \end{enumerate}
    Dado que $d$ en $X$ cumple las cuatro propiedades anteriores, concluimos que $d$ es una métrica en $X$. A esta métrica se le conoce como \emph{métrica discreta}.
\end{examplebox}

\begin{definicion}{}{}
    Un \emph{espacio métrico} es un conjunto $X$ dotado de una función llamada distancia o métrica. Un espacio métrico se denota como $(X, d)$.
\end{definicion}

\begin{examplebox}{}{}
    Consideremos el conjunto $C[a, b])$, que contiene todas las funciones continuas $f: [a, b] \longrightarrow \RR$. Definimos la siguiente métrica en este espacio:
    $$d(f, g) = \int_{a}^{b} |f(x) - g(x)|  dx.$$
    Esta métrica, convierte $C[a, b]$ en un espacio métrico.

    \tcblower
    \demostracion Verifiquemos las propiedades de la métrica:
    \begin{enumerate}[label=\roman*), topsep=6pt, itemsep=0pt]
        \item Como $|f(x) - g(x)| \geq 0$ para todo $x \in [a, b]$, la integral $d(f, g) \geq 0$.
        \item Si $f = g$, entonces $|f(x) - g(x)| = 0$ para todo $x \in [a, b]$, de donde se sigue que $d(f, g) = 0$. Recíprocamente, si $d(f, g) = 0$, como $|f(x) - g(x)|$ es continua y no negativa, se deduce que $|f(x) - g(x)| = 0$ para todo $x \in [a, b]$. Por tanto, $f = g$.
        \item Es evidente, ya que se hereda de saber que $|f(x) - g(x)| = |g(x) - f(x)|$.
        \item Sean $f$, $g$, $h \in C[a, b]$,
        \begin{align*}
            d(f, h) & = \int_{a}^{b} |f(x) - h(x)| dx \\
            & \leq \int_{a}^{b} \left( |f(x) - g(x)| + |g(x) - h(x)| \right) dx \\
            & = d(f, g) + d(g, h)
        \end{align*}
    \end{enumerate}
    Por lo tanto, $(C[a, b], d)$ es un espacio métrico.
\end{examplebox}

\newpage

\begin{examplebox}{}{}
    Sea $\RR[n]$ con la métrica $d_2$. Para $\mathbf{x} = (x_1, x_2, \dots, x_n)$ e $\mathbf{y} = (y_1, y_2, \dots, y_n)$ en $\RR[n]$, se define:
    $$d_2(\mathbf{x}, \mathbf{y}) = \sqrt{(x_1 - y_1)^2 + (x_2 - y_2)^2 + \cdots + (x_n - y_n)^2}.$$
    Esta métrica corresponde a la norma $\| \phantom{x} \|_2$, como se demostró previamente. Por tanto, $(\RR[n], d_2)$ es un espacio métrico.
\end{examplebox}

Es importante observar que todo subconjunto $Y$ de un espacio métrico $X$ es, a su vez, un espacio métrico, utilizando la misma función de distancia definida en $X$. Esto se debe a que las propiedades que caracterizan a una métrica en $X$, es decir, las condiciones (i) a (iv) de la definición \ref{definicion:metrica}, se mantienen válidas cuando se restringen a un subconjunto $Y$ de $X$. En particular, si $d$ es la función de distancia en $X$ y se cumplen las condiciones para puntos $\mathbf{x}$, $\mathbf{y}$, $\mathbf{z} \in X$, estas mismas condiciones se cumplen cuando se considera que $\mathbf{x}$, $\mathbf{y}$, $\mathbf{z}$ pertenecen a $Y$.

\begin{definicion}{}{}
    Dado un espacio métrico $(X, d)$, se define la \emph{bola abierta} y la \emph{bola cerrada} de centro $\mathbf{x}_0 \in X$ y radio $r > 0$ respectivamente como 
    $$B(\mathbf{x}_0, r) = \{ \mathbf{x} \in X \mid d(\mathbf{x}, \mathbf{x}_0) < r \} \quad \text{ y } \quad \overline{B}(\mathbf{x}_0, r) = \{ \mathbf{x} \in X \mid d(\mathbf{x}, \mathbf{x}_0) \leq r \}.$$
\end{definicion}

\sideFigure[\label{fig:bola1}]{
    \begin{tikzpicture}
        \node at (0,3.5) {~};
        \filldraw[cw2] (0,1) -- (-1,0) -- (0,-1) -- (1,0) -- cycle;
        \draw[dash pattern=on 3pt off 3pt] (0,1) -- (-1,0) -- (0,-1) -- (1,0) -- cycle;
        \draw[latex-] (0.55,0.55) -- (1,1) node[above,xshift=0.7cm] {$x + y = 1$};
        \draw[latex-] (-0.55,0.55) -- (-1,1) node[above,xshift=-0.7cm] {$- x + y = 1$};
        \draw[latex-] (-0.55,-0.55) -- (-1,-1) node[below,xshift=-0.7cm] {$- x - y = 1$};
        \draw[latex-] (0.55,-0.55) -- (1,-1) node[below,xshift=0.7cm] {$x - y = 1$};
        \draw[-Stealth] (-2.5,0) -- (2.5,0) node[below left] {$x$};
        \draw[-Stealth] (0,-2.5) -- (0,2.5) node[below left] {$y$};
    \end{tikzpicture}
}

\begin{examplebox}{}{}
    Represente la bola $B(\mathbf{0}, 1)$, en $\RR[2]$, utilizando la métrica $d_1$.

    \tcblower
    \solucion Para trazar la bola anterior en el gráfico, es necesario observar que
    \begin{align*}
        (x, y) \in B( \mathbf{0}, 1) & \Longleftrightarrow d_1 \big( (x, y), (0, 0) \big) < 1 \\
        & \Longleftrightarrow \| (x, y) - (0, 0) \|_1 < 1 \\
        & \Longleftrightarrow \| (x, y) \|_1 < 1 \\
        & \Longleftrightarrow |x| + |y| < 1
    \end{align*}
    La desigualdad $|x| + |y| < 1$ describe un rombo (o diamante) centrado en el origen $(0, 0)$ con vértices en los puntos: $(1, 0)$, $(-1, 0)$, $(0, 1)$, $(0, -1)$. Sin embargo, como la desigualdad es estricta, la bola excluye el borde y solo incluye los puntos interiores. Los vértices se obtienen al considerar los casos extremos donde una coordenada es máxima mientras la otra es cero.
    \begin{itemize}[topsep=6pt, itemsep=0pt]
        \item Si $y = 0$, entonces $|x| < 1$ y por consiguiente, $x \in (-1, 1)$.
        \item Si $x = 0$, entonces $|y| < 1$ y por consiguiente, $y \in (-1, 1)$.
    \end{itemize}
    Cada arista del rombo corresponde a una ecuación lineal: $x + y = 1$, $-x + y = 1$, $-x - y = 1$, $x - y = 1$ para cada cuadrante respectivamente. Por lo tanto, la representación gráfica de esta bola está dada por la figura \ref{fig:bola1}.
\end{examplebox}

\sideFigure[\label{fig:bola3}]{\vfill
    \begin{tikzpicture}
        \node at (0,3.5) {~};
        \filldraw[cw2] (0,0) circle (1cm);
        \draw[dash pattern=on 3pt off 3pt] (0,0) circle (1cm);
        \draw[latex-] (0.6,0.9) -- (1.3,1.5) node[above] {$x^2 + y^2 = 1$};
        \draw[-Stealth] (-2.5,0) -- (2.5,0) node[below left] {$x$};
        \draw[-Stealth] (0,-2.5) -- (0,2.5) node[below left] {$y$};
    \end{tikzpicture}
}

\begin{examplebox}{}{}
    Represente la bola $B(\mathbf{0}, 1)$, en $\RR[2]$, utilizando la métrica $d_2$.

    \tcblower
    \solucion Para trazar la bola anterior en el gráfico, es necesario observar que
    \begin{align*}
        (x, y) \in B(\mathbf{0}, 1) & \Longleftrightarrow d_2 \big( (x, y), (0, 0) \big) < 1 \\
        & \Longleftrightarrow \| (x, y) - (0, 0) \|_2 < 1 \\
        & \Longleftrightarrow \| (x, y) \|_2 < 1 \\
        & \Longleftrightarrow \sqrt{x^2 + y^2} < 1
    \end{align*}
    Esta desigualdad describe un círculo de radio $1$ centrado en el origen $(0, 0)$. Todos los puntos $(x, y)$ que se encuentran dentro de esta región cumplen que la distancia euclidiana al origen es estrictamente menor que $1$, es decir, que están situados dentro de la bola de radio 1. El límite de la bola (circunferencia) está dado por $x^2 + y^2 = 1$, que corresponde a los puntos a distancia exactamente $1$ del origen. Por lo tanto, la representación gráfica de esta bola está dada por la figura \ref{fig:bola3}.
\end{examplebox}

\newpage
\sideFigure[\label{fig:bola2}]{
    \begin{tikzpicture}
        \node at (0,3.5) {~};
        \filldraw[cw2] (-1,-1) rectangle (1,1);
        \draw[dash pattern=on 3pt off 3pt] (-1,-1) rectangle (1,1);
        \draw[latex-] (0.5,1.05) -- (0.75,1.75) node[right] {$y = 1$};
        \draw[latex-] (-0.5,-1.05) -- (-0.75,-1.75) node[left] {$y = - 1$};
        \draw[latex-] (1.05,-0.5) -- (1.75,-0.75) node[below] {$x = 1$};
        \draw[latex-] (-1.05,0.5) -- (-1.75,0.75) node[above] {$x = - 1$};
        \draw[-Stealth] (-2.5,0) -- (2.5,0) node[below left] {$x$};
        \draw[-Stealth] (0,-2.5) -- (0,2.5) node[below left] {$y$};
    \end{tikzpicture}
}

\begin{examplebox}{}{}
    Represente la bola $B(\mathbf{0}, 1)$, en $\RR[2]$, utilizando la métrica $d_{\infty}$.

    \tcblower
    \solucion Para trazar la bola anterior en el gráfico, es necesario observar que
    \begin{align*}
        (x, y) \in B(\mathbf{0}, 1) & \Longleftrightarrow d_{\infty} \big( (x, y), (0, 0) \big) < 1 \\
        & \Longleftrightarrow \| (x, y) - (0, 0) \|_{\infty} < 1 \\
        & \Longleftrightarrow \| (x, y) \|_{\infty} < 1 \\
        & \Longleftrightarrow \max \left\{ |x| + |y| \right\} < 1
    \end{align*}
    Esta desigualdad describe un cuadrado centrado en el origen $(0, 0)$, con lados paralelos a los ejes coordenados y longitud de lado $2$. Los vértices del cuadrado están en los puntos: $(1, 1)$, $(1, -1)$, $(-1, -1)$, $(-1, 1)$. Sin embargo, como la desigualdad es estricta, la bola excluye el borde y solo incluye los puntos interiores. Esto es por los siguientes dos casos:
    \begin{enumerate}[label=\roman*), topsep=6pt, itemsep=0pt]
        \item Si $\max \left\{ |x|, |y| \right\} = |y|$, entonces $|x| \leq |y| = 1$; por consiguiente, $y = \pm 1$. Si $y = 1$, entonces $|x| \leq 1$, lo que implica que $-1 \leq x \leq 1$. De manera similar, si $y = -1$, entonces $|x| \leq -1$, lo que implica que $-1 \leq x \leq 1$.
        \item Si $\max \left\{ |x|, |y| \right\} = |x|$, entonces $|y| \leq |x| = 1$; por consiguiente, $x = \pm 1$. Si $x = 1$, entonces $|y| \leq 1$, lo que implica que $-1 \leq y \leq 1$. De manera similar, si $x = -1$, entonces $|y| \leq -1$, lo que implica que $-1 \leq y \leq 1$.
    \end{enumerate}
    Por lo tanto, la representación gráfica de esta bola está dada por la figura \ref{fig:bola2}.
\end{examplebox}

Observemos que en un espacio métrico $(X, d)$, cualquier bola $B(\mathbf{x}_0, r)$ de centro $\mathbf{x}_0 \in X$ y radio $r > 0$ puede obtenerse a partir de la \emph{bola unitaria} $B(\mathbf{0}, 1)$, la bola centrada en el origen con radio $1$, mediante dos operaciones fundamentales:
\begin{itemize}
    \item Homotecia: Ajustar el radio de la bola unitaria para cambiar su “tamaño”. Si consideramos la bola unitaria $B(\mathbf{0}, 1)$, multiplicar cada distancia por un factor $r > 0$ transforma la bola en $B(\mathbf{0}, r)$. Es decir,
    $$B(\mathbf{0}, r) = \left\{ \mathbf{x} \in X \mid d(\mathbf{x}, \mathbf{0}) < r \right\} = r \cdot B(\mathbf{0}, 1),$$
    donde $r \cdot B(\mathbf{0}, 1)$ denota escalar todos los puntos de la bola unitaria por $r$.  
    \item Traslación: Desplazar el centro de la bola desde el origen $\mathbf{0}$ hasta el punto deseado $\mathbf{x}_0$. Para mover el centro de $\mathbf{0}$ a $\mathbf{x}_0$, se suma $\mathbf{x}_0$ a cada punto de la bola escalada. Es decir,
    $$B(\mathbf{x}_0, r) = \mathbf{x}_0 + B(\mathbf{0}, r) = \left\{ \mathbf{x}_0 + \mathbf{y} \mid \mathbf{y} \in B(\mathbf{0}, r) \right\}.$$
\end{itemize}

Por ejemplo, veamos que la bola $B\big((3, 2), 2\big)$ en $\RR[2]$ bajo la métrica $d_2$, se puede construir mediante la bola unitaria.
\begin{itemize}
    \item Homotecia: La bola unitaria $B(\mathbf{0}, 1)$ se escala multiplicando todas las distancias por el radio $r = 2$. Es decir,
    $$B(\mathbf{0}, 2) = 2 \cdot B(\mathbf{0}, 1) = \left\{ \mathbf{x} \in \RR[2] \mid d_2(\mathbf{x}, \mathbf{0}) < 2 \right\}.$$
    \item Traslación: Se desplaza la bola escalada $B(\mathbf{0}, 2)$ al punto $(3, 2)$ sumando el vector $(3, 2)$ a todos sus puntos. Es decir,
    $$B\big((3, 2), 2\big) = (3, 2) + B(\mathbf{0}, 2).$$
\end{itemize}
Si aplicáramos la definición de bola, entonces tenemos que
\begin{align*}
    (x, y) \in B\big((3, 2), 2\big) & \Longleftrightarrow d_2 \big( (x, y), (3, 2) \big) < 2 \\
    & \Longleftrightarrow \| (x, y) - (3, 2) \|_2 < 2 \\
    & \Longleftrightarrow \| (x - 3, y - 2) \|_2 < 2 \\
    & \Longleftrightarrow \sqrt{(x - 3)^2 + (y - 2)^2} < 2
\end{align*}
lo que es equivalente a lo que obtuvimos anteriormente.

\newpage
La representación gráfica de dicha bola está dada como:
\begin{figure}[h!]
    \centering
    \begin{tikzpicture}
        \filldraw[cw2] (0,0) circle (1);
        \filldraw[cw2] (3,2) circle (2);
        \draw[dashed] (0,0) circle (1);
        \draw[dashed] (3,2) circle (2);
        \draw[latex-] (-0.6,0.9) -- (-1.3,1.5) node[above] {$x^2 + y^2 = 1$};
        \draw[latex-] (4.2,3.8) -- (5.1,4.4) node[above] {$(x - 3)^2 + (x - 2)^2 = 4$};
        \draw[-stealth] (-1.5,0) -- (6,0) node[below left] {$x$};
        \draw[-stealth] (0,-1.5) -- (0,5) node[below left] {$y$};
    \end{tikzpicture}
    \caption{Bola unitaria y bola $B\big((3, 2), 2\big)$ en $\RR[2]$ con la métrica $d_2$}
\end{figure}

Llegamos ahora a los conceptos que son fundamentales para el estudio de las propiedades topológicas.

\begin{definicion}{}{}
    Sea $(X, d)$ un espacio métrico y sea $A$ un subconjunto de $X$. Decimos que $\mathbf{x} \in A$ es un \emph{punto interior} de $A$ si existe $r > 0$ tal que
    $$B(\mathbf{x}, r) \subseteq A.$$
    Al conjunto de puntos interiores de $A$ lo denotaremos por $\operatorname{int}(A)$.
\end{definicion}

Notemos que para todo $A \subseteq X$, $\operatorname{int} (A) \subseteq A$. Además, $\mathbf{x} \in A$ no es punto interior de $A$ si para todo $r > 0$, $B(\mathbf{x}, r) \nsubseteq A$.

\begin{examplebox}{}{}
    Sea el espacio métrico $(\RR[2], d_2)$, es decir, para $\mathbf{p} = (p_1, p_2)$ y $\mathbf{q} = (q_1, q_2)$ en $\RR[2]$, la distancia está dada por
    $$d_2(\mathbf{p}, \mathbf{q}) = \sqrt{(p_1 - q_1)^2 + (p_2 - q_2)^2}.$$
    Ahora, consideremos el conjunto
    $$A = \left\{(x_1, x_2) \in \RR[2] \mid x_1^2 + x_2^2 < 1\right\},$$
    Queremos encontrar un radio $r > 0$ tal que la bola abierta de radio $r$ centrada en $\mathbf{x}$ esté completamente contenida en $A$. Podemos tomar
    $$r = 1 - \sqrt{x_1^2 + x_2^2}.$$
    Dado que $\mathbf{x}$ está estrictamente dentro del círculo (no en la frontera), se tiene que $r > 0$. Además, para cualquier punto $(y_1, y_2) \in B(\mathbf{x}, r)$, se cumple
    $$\sqrt{(y_1 - x_1)^2 + (y_2 - x_2)^2} < r.$$
    La desigualdad del triángulo en $\RR[2]$ nos dice que para cualquier $\mathbf{x} = (x_1, x_2)$ y $\mathbf{y} = (y_1, y_2) \in \RR[2]$, se cumple que $d( \mathbf{0}, \mathbf{y} ) \leq d( \mathbf{0}, \mathbf{x} ) + d( \mathbf{x}, \mathbf{y} )$. Escribiéndolo explícitamente en términos de la norma euclidiana,
    \begin{align*}
        \sqrt{y_1^2 + y_2^2} & \leq \sqrt{x_1^2 + x_2^2} + \sqrt{(y_1 - x_1)^2 + (y_2 - x_2)^2} \\
        & < \sqrt{x_1^2 + x_2^2} + r = 1
    \end{align*}
    Esto muestra que $(y_1, y_2) \in A$, lo que implica que $B(\mathbf{x}, r) \subseteq A$, es decir, $\mathbf{x}$ es un punto interior de $A$.
\end{examplebox}

\newpage

\begin{examplebox}{}{}
    Consideremos el espacio métrico $(\RR, d)$ con la métrica usual $d(x, y) = |x - y|$. Sea $A = [0, 2]$, un intervalo cerrado en $\RR$. Para cualquier $x \in (0, 2)$, existe $r > 0$ tal que $B(x, r) \subseteq A$. Por ejemplo, si $x = 1$, tomamos $r = 0.5$. Entonces
    $$B(1, 0.5) = (0.5, 1.5) \subseteq [0, 2].$$
    En general, para $x \in (0, 2)$, basta elegir $r = \min \{x, 2 - x\}$, lo que garantiza que la bola $B(x, r)$ no toque los extremos $0$ o $2$. Esto se sigue de los siguientes casos:
    \begin{itemize}[topsep=6pt, itemsep=0pt]
        \item En $x = 0$, cualquier bola $B(0, r) = (-r, r)$ incluye números negativos, lo que conlleva a que no pertenezca a $A$. Por tanto, $0$ no es punto interior.
        \item En $x = 2$, cualquier bola $B(2, r) = (2 - r, 2 + r)$, lo que conlleva a que no pertenezca a $A$. Por tanto, $2$ tampoco es punto interior
    \end{itemize}
    El conjunto de puntos interiores de $A$ es $\operatorname{int}(A) = (0, 2)$.
\end{examplebox}

Este ejemplo muestra en general, que en $\RR$, el interior de un intervalo cerrado $A = [a, b]$ es el intervalo abierto $(a, b)$, ya que los extremos no tienen “espacio” alrededor dentro de $A$.

\begin{definicion}{}{}
    Sea $(X, d)$ un espacio métrico y sea $A$ un subconjunto de $X$. Decimos que $A$ es un \emph{conjunto abierto} en $X$ si $A = \operatorname{int}(A)$, es decir, si para todo $\mathbf{x} \in A$ existe $r > 0$ tal que $B(\mathbf{x}, r) \subseteq A$.
\end{definicion}

Observemos que $\varnothing$ es un conjunto abierto en $X$ por vacuidad ya que si $\varnothing$ no es un conjunto abierto en $X$ debe existir un punto en $D$ que no sea interior pero eso es imposible, por tanto $D$ es un conjunto abierto en $X$.

\sideFigure[\label{fig:anillo_abierto}]{
\begin{tikzpicture}
    \node at (0,3.75) {~};
    \filldraw[cw2] (0,0) circle (2);
    \filldraw[white] (0,0) circle (1);
    \draw[dashed] (0,0) circle (2);
    \draw[dashed] (0,0) circle (1);
    \draw[-stealth] (-2.5,0) -- (2.5,0) node[below left] {$x$};
    \draw[-stealth] (0,-2.5) -- (0,2.5) node[below left] {$y$};
    \node[above right] at (0,-1) {$\sqrt{a}$};
    \node[above right] at (0,-2) {$\sqrt{b}$};
\end{tikzpicture}
}

\begin{examplebox}{}{}
    Sea $(\RR[2], d_2)$ un espacio métrico. Demuestre que el siguiente conjunto es abierto:
    $$D = \left\{ (x, y) \in \RR[2] \mid a < x^2 + y^2 < b \right\}.$$

    \tcblower
    \demostracion Observemos que el conjunto $D$ representa un anillo abierto centrado en el origen, excluyendo las circunferencias de radios $\sqrt{a}$ y $\sqrt{b}$ (vea la figura \ref{fig:anillo_abierto}). Ahora bien, sea $(x_0, y_0) \in D$, lo que implica que $a < x_0^2 + y_0^2 < b$. Queremos encontrar $r > 0$ tal que $B\big((x_0, y_0), r\big) \subseteq D$. Definamos $\rho_0 = x_0^2 + y_0^2$ y tomemos
    $$r = \min \{\rho_0 - a, b - \rho_0\} > 0.$$
    Si $(x, y) \in B\big((x_0, y_0), r\big)$, significa que
    $$\sqrt{(x - x_0)^2 + (y - y_0)^2} < r.$$
    Por la desigualdad triangular,
    \begin{align*}
        \left|x^2 + y^2 - \rho_0\right| & = \left|x^2 + y^2 - x_0^2 - y_0^2\right| \\
        & = |(x - x_0)(x + x_0) + (y - y_0)(y + y_0)| \\
        & \leq |x - x_0| |x + x_0| + |y - y_0| |y + y_0| \\
        & \leq \sqrt{(x - x_0)^2 + (y - y_0)^2} < r
    \end{align*}
    Esto implica que
    $$\rho_0 - r < x^2 + y^2 < \rho_0 + r.$$
    Dado que $r \leq \rho_0 - a$ y $r \leq b - \rho_0$, tenemos
    $$a < \rho_0 - r < x^2 + y^2 < \rho_0 + r < b.$$
    Por lo tanto, todo punto en $B\big((x_0, y_0), r\big)$ también pertenece a $D$, lo que implica que $B\big((x_0, y_0), r\big) \subseteq D$. Por consiguiente, $D$ es un conjunto abierto.
\end{examplebox}

\newpage
\sideFigure[Las bolas abiertas, son conjuntos abiertos\label{fig:bolas_abiertas}]{
\begin{tikzpicture}
    \node at (0,3.5) {~};
    \coordinate (x);
    \coordinate[below right=1cm and 1cm of x] (p);
    \filldraw[cw2] (x) circle (2.5);
    \filldraw[white] (p) circle (1);
    \draw[dashed] (x) node[above,yshift=3pt] {$\mathbf{x}$} circle (2.5);
    \draw[dashed] (p) node[above,yshift=3pt] {$\mathbf{p}$} circle (1);
    \filldraw (x) circle (1.75pt);
    \filldraw (p) circle (1.75pt);
    \draw (x) -- node[midway,above] {$r$} ($(x) + (215:2.5)$);
    \draw (p) -- node[midway,above] {$s$} ($(p) + (315:1)$);
\end{tikzpicture}
}

\begin{theorem}{}{}
    En un espacio métrico, cada bola abierta es un conjunto abierto.

    \tcblower
    \demostracion Sea la bola abierta $B(\mathbf{x}, r)$ y veamos que si $\mathbf{p} \in B(\mathbf{x}, r)$, existe un $s > 0$ tal que $B(\mathbf{p},s) \subseteq B(\mathbf{x}, r)$. En efecto, tomemos $s = r - d(\mathbf{x},\mathbf{p}) > 0$, y comprobemos que si $\mathbf{q} \in B(\mathbf{p}, s)$, entonces $\mathbf{q} \in B(\mathbf{x}, r)$. Tenemos que $d(\mathbf{p}, \mathbf{q}) < s$ y según la desigualdad triangular
    \begin{align*}
        d(\mathbf{x},\mathbf{q}) & \leq d(\mathbf{x},\mathbf{p}) + d(\mathbf{p},\mathbf{q}) \\
        & < d(\mathbf{x},\mathbf{p}) + s = r
    \end{align*}
    lo que significa que $\mathbf{q} \in B(\mathbf{x}, r)$ y por tanto, $B(\mathbf{p}, s) \subseteq B(\mathbf{x},r)$ (vea la figura \ref{fig:bolas_abiertas}).
\end{theorem}

Este hecho, en conjunto con la propiedad de separación de puntos que exploramos a continuación, forma parte de las características fundamentales de los espacios métricos. La propiedad de Hausdorff, que enuncia la existencia de bolas abiertas disjuntas alrededor de dos puntos distintos, es crucial para entender cómo se comportan las sucesiones y los conjuntos en estos espacios, especialmente en lo que respecta a la unicidad de los límites de sucesiones.

\begin{prop}{}{Hausdorff}
    \TituloBox{Propiedad de Hausdorff:} Sea $(X, d)$ un espacio métrico y dos puntos distintos $\mathbf{x}$, $\mathbf{y} \in X $. Entonces existen $r_1, r_2 > 0$ tales que
    $$B(\mathbf{x}, r_1) \cap B(\mathbf{y}, r_2) = \varnothing.$$

    \tcblower
    \demostracion Sea $r = d(\mathbf{x}, \mathbf{y})$ y sea $s = r/2$, entonces las bolas $B(\mathbf{x}, s)$ y $B(\mathbf{y}, s)$ abiertas, tienen intersección vacía. En efecto, veamos que ningún punto de la primera puede estar en la segunda. Si $\mathbf{z} \in B(\mathbf{x}, s)$, entonces, por la desigualdad del triángulo
    \begin{align*}
        d(\mathbf{z}, \mathbf{y}) & \geq d(\mathbf{x}, \mathbf{y}) - d(\mathbf{z}, \mathbf{x}) \\
        & = r - d(\mathbf{z}, \mathbf{x}) \\
        & > r - s = r/2
    \end{align*}
    con lo que $\mathbf{z} \notin B(\mathbf{y}, s)$. Para la otra bola se hace de la misma forma.
\end{prop}

\sideFigure[La intersección de bolas abiertas es abierto\label{fig:interseccion_bolas}]{
\begin{tikzpicture}
    \node at (0,3) {~};
    \coordinate (a);
    \coordinate[below right=0.5cm and 1.9cm of x] (b);
    \coordinate (x) at ($(a)!0.56!(b)$);
    \filldraw[cw2] (a) circle (1.75);
    \filldraw[cw1!50] (b) circle (1.5);
    \filldraw[white] (x) circle (0.6);
    \draw[dashed] (a) node[above,yshift=3pt] {$\mathbf{a}$} circle (1.75);
    \draw[dashed] (b) node[above,yshift=3pt] {$\mathbf{b}$} circle (1.5);
    \draw[dashed] (x) node[below,yshift=-3pt] {$\mathbf{x}$} circle (0.6);
    \filldraw (a) circle (1.75pt);
    \filldraw (b) circle (1.75pt);
    \filldraw (x) circle (1.75pt);
    \draw (a) -- node[midway,above] {$r$} ($(a) + (150:1.75)$);
    \draw (b) -- node[midway,above] {$s$} ($(b) + (315:1.5)$);
    \draw (x) -- node[midway,right] {$\delta$} ($(x) + (90:0.6)$);
\end{tikzpicture}
}

\begin{lemma}{}{interseccion_bolas}
    La intersección de dos bolas abiertas en un espacio métrico $(X, d)$, es un abierto.

    \tcblower
    \demostracion Si la intersección de ambas bolas es vacía, no hay nada que probar, pues $\varnothing$ es abierto. Supongamos entonces que $x \in B(\mathbf{a}, r) \cap B(\mathbf{b}, s)$ y veamos que tal intersección contiene alguna bola centrada en $\mathbf{x}$. Se cumple que $d(\mathbf{x}, \mathbf{a}) < r$ y $d(\mathbf{x}, \mathbf{b}) < s$; así que tomemos $\delta < \min\{r - d(\mathbf{x}, \mathbf{a}), s - d(\mathbf{x}, \mathbf{b})\}$ y comprobemos que $B(\mathbf{x}, \delta) \subseteq B(\mathbf{a}, r) \cap B(\mathbf{b}, s)$ (véase la figura \ref{fig:interseccion_bolas}). En efecto, si $\mathbf{y} \in B(\mathbf{x}, \delta)$, entonces
    \begin{align*}
        d(\mathbf{y}, \mathbf{a}) & \leq d(\mathbf{y}, \mathbf{x}) + d(\mathbf{x}, \mathbf{a}) \\
        & < \delta + d(\mathbf{x}, \mathbf{a}) \\
        & < r - d(\mathbf{x}, \mathbf{a}) + d(\mathbf{x}, \mathbf{a}) = r
    \end{align*}
    y por tanto $\mathbf{y} \in B(\mathbf{a}, r)$. De manera análoga, se demuestra que $B(\mathbf{x}, \delta) \subseteq B(\mathbf{b}, s)$. Con esto hemos probado que la intersección de las dos bolas contiene una bola centrada en cada uno de sus puntos y, por lo tanto es un abierto.
\end{lemma}

\begin{theorem}{}{topologia}
    Sea $(X, d)$ un espacio métrico. Entonces se cumplen las siguientes propiedades:
    \begin{enumerate}[label=\roman*), topsep=6pt, itemsep=0pt]
        \item $X$ y $\varnothing$ son abiertos.
        \item La unión de una familia cualquiera de conjuntos abiertos, es abierto.
        \item La intersección de una colección finita de conjuntos abiertos, es abierto.
    \end{enumerate}
    \newpage
    \demostracion
    \begin{enumerate}[label=\roman*), topsep=6pt, itemsep=0pt]
        \item Es inmediato.
        \item Sea $\{A_i\}_{i \in I}$ una familia cualquiera de subconjuntos abiertos en $X$, donde $I$ es un conjunto de índices (no necesariamente finito). Consideremos el conjunto abierto $\displaystyle A = \bigcup_{i \in I} A_i$. Tomemos un $\mathbf{x} \in A$. Por definición de unión, existe al menos un índice $i \in I$ tal que $\mathbf{x} \in A_i$. Como $A_i$ es abierto, existe un $r > 0$ tal que la bola abierta $B(\mathbf{x}, r) \subseteq A_i$. Dado que $A_i \subseteq A$, tenemos que $B(\mathbf{x}, r) \subseteq A_i \subseteq A$, y por lo tanto $B(\mathbf{x}, r) \subseteq A$. Esto es cierto para cualquier $\mathbf{x} \in A$, lo que significa que $A$ satisface la definición de un conjunto abierto.
        \item Si la intersección es vacía no hay nada que probar. Supongamos entonces, que $A_1$ y $A_2$ son dos conjuntos abiertos cuya intersección es no vacía. Si $\mathbf{x} \in A_1 \cap A_2$, existen $r_1$, $r_2 > 0$ de modo $B(\mathbf{x}, r_1) \subseteq A_1$ y $B(\mathbf{x}, r_2) \subseteq A_2$; entonces según el lema \ref{lemma:interseccion_bolas} hay una bola centrada en $x$ contenida en la intersección de ambas bolas, lo que implica que dicha bola también está en $A_1 \cap A_2$ y que este último conjunto es abierto. Mediante un sencillo proceso de inducción se prueba que la intersección de cualquier familia finita de abiertos es un abierto.
    \end{enumerate}
\end{theorem}

A la familia de todos los conjuntos abiertos de un espacio métrico $(X, d)$ se le llama \emph{topología} asociada a la métrica $d$ y la designaremos mediante $\mathcalm{T}_d$, o simplemente $\mathcalm{T}$ si no hay ambigüedad respecto de la métrica.

En general, si tenemos un conjunto $X$, a cualquier familia de subconjuntos de $X$ que verifica las tres condiciones del teorema \ref{theorem:topologia} se le llama \emph{topología} sobre $X$. En este curso, nos limitaremos a estudiar topologías asociadas a espacios métricos aunque hay espacios topológicos que no son métricos, como se muestra en el siguiente ejemplo.

\newpage

\section{Sucesiones}

En esta sección vamos a estudiar el concepto de sucesión en un espacio métrico. Estos subconjuntos juegan un papel importante en la topología de los espacios métricos.

\begin{definicion}{}{}
    Sea $(X, d)$ un espacio métrico. Una \emph{sucesión} en $X$ es un subconjunto de $X$ definido mediante una función, de tal modo que
    \begin{alignat*}{2}
        x &: & \quad \NN & \longrightarrow X \\
        & & n & \longmapsto \mathbf{x}_n
    \end{alignat*}
    Denotaremos a la sucesión mediante $(\mathbf{x}_n)_{n \in \NN}$, o $\{\mathbf{x}_n\}_{n = 1}^{\infty}$ o simplemente $(\mathbf{x}_n)$; y a los elementos de la sucesión les llamaremos \emph{términos}.
\end{definicion}

La función que define la sucesión no ha de ser necesariamente inyectiva, lo que significa que puede haber términos repetidos en una sucesión; por ejemplo $\left\{(-1)^n\right\}_{n = 1}^{\infty}$ es la sucesión $\{-1, 1, -1, 1, \dots \}$.

\begin{definicion}{}{}
    Sea $(X, d)$ un espacio métrico y $\{\mathbf{x}_n\}_{n = 1}^{\infty}$ una sucesión de puntos de $X$. Diremos que $\{\mathbf{x}_n\}_{n = 1}^{\infty}$ converge a $\mathbf{x}$ en $(X, d)$, y lo denotaremos por $\mathbf{x}_n \to \mathbf{x}$ o $\displaystyle \lim_{n \to \infty} \mathbf{x}_n = \mathbf{x}$, si para todo $\varepsilon > 0$, existe $n_0 \in \NN$, tal que si $n \geq n_0$, entonces $d(\mathbf{x}_n, \mathbf{x}) < \varepsilon$.
\end{definicion}

En otras palabras si para toda bola $B(\mathbf{x}, \varepsilon)$, existe $n_0 \in \NN$, tal que si $n \geq n_0$, entonces $\mathbf{x}_n \in B(\mathbf{x}, \varepsilon)$. En este caso se dice que la sucesión es convergente hacia el punto $\mathbf{x}$, o que $\mathbf{x}$ es el límite de la sucesión.

\begin{examplebox}{}{}
    El concepto de convergencia que acabamos de definir coincide con el ya conocido de convergencia en $\RR$ con el valor absoluto, es decir, en la topología usual. Recordemos que una sucesión $\{x_n\}_{n = 1}^{\infty}$ converge a $x \in \RR$ si para todo $\varepsilon > 0$, existe $n_0$ tal que si $n > n_0$, entonces $|x_n - x| < \varepsilon$. Fijémonos que si $|x_n - x| < \varepsilon$, entonces
    $$-\varepsilon < x_n - x < \varepsilon \quad \text{ y } \quad x - \varepsilon < x_n < x + \varepsilon,$$
    lo que conlleva a $x_n \in (x - \varepsilon, x + \varepsilon)$ y tenemos la definición en términos de bolas.
\end{examplebox}

\begin{examplebox}{}{}
    Si $(X, d)$ es un espacio discreto una sucesión $\{\mathbf{x}_n\}_{n = 1}^{\infty}$ converge a un punto $\mathbf{x}$ si, y sólo si es constante igual a $\mathbf{x}$ a partir de un término (a estas sucesiones se les llama de cola constante), ya que las bolas de centro $\mathbf{x}$ y radio menor que $1$ coinciden con el conjunto unipuntual $\{\mathbf{x}\}$.
\end{examplebox}

\begin{theorem}{}{}
    Si $\{\mathbf{x}_n\}_{n = 1}^{\infty}$ es una sucesión convergente en un espacio métrico $(X, d)$, su límite es único.

    \tcblower
    \demostracion Supongamos que $\{\mathbf{x}_n\}_{n = 1}^{\infty}$ tiene dos límites distintos $\mathbf{x}$ e $\mathbf{y}$, tal que $\mathbf{x} \neq \mathbf{y}$. Según la proposición \ref{prop:Hausdorff}, $X$ es un espacio de Hausdorff y, por tanto existe $r > 0$ tal que $B(\mathbf{x}, r) \cap B(\mathbf{y}, r) = \varnothing$. Por otro lado, como $\{\mathbf{x}_n\}_{n = 1}^{\infty}$ converge a $\mathbf{x}$, tenemos que dado $r > 0$, existe $n_1$ tal que si $n \geq n_1$, entonces $\mathbf{x}_n \in B(\mathbf{x}, r)$; además como $\{\mathbf{x}_n\}_{n = 1}^{\infty}$ converge a $\mathbf{y}$, dado $r > 0$, existe $n_2$ tal que si $n \geq n_2$, entonces $\mathbf{x}_n \in B(\mathbf{y}, r)$; si tomamos $n \geq n_1$ y a la vez $n \geq n_2$ se verifican ambas condiciones a la vez y $\mathbf{x}_n \in B(\mathbf{x}, r)$ y $\mathbf{x}_n \in B(\mathbf{y}, r)$, lo que contradice que la intersección de estas dos bolas es vacía.
\end{theorem}

\newpage

\section{Ejercicios del Capítulo 4}

\begin{enumerate}
    \item Si $\mathbf{x} = (x_1, x_2, \dots, x_n)$, demostrar:
    \begin{enumerate}
        \item $|x_i| \leq \| \mathbf{x} \| \leq \sqrt{n} \max \{ |x_1|, |x_2|, \dots, |x_n|\}$,
        \item $\| \mathbf{x} \|_{\infty} \leq \| \mathbf{x} \| \leq \sqrt{n} \| \mathbf{x} \|_{\infty}$.
    \end{enumerate}
    En donde: $\| \mathbf{x} \| = \sqrt{x_1^2 + x_2^2 + \cdots + x_n^2}$.
    \item Graficar los siguientes conjuntos:
    \begin{enumerate}
        \item $B_{d_1}(\mathbf{0}, 1) = \{ \mathbf{x} \in \RR[2] : \| \mathbf{x} \|_1 \leq 1 \}$,
        \item $B_{d_2}(\mathbf{0}, 1) = \{ \mathbf{x} \in \RR[2] : \| \mathbf{x} \|_2 \leq 1 \}$,
        \item $B_{d_{\infty}}(\mathbf{0}, 1) = \{ \mathbf{x} \in \RR[2] : \| \mathbf{x} \|_{\infty} \leq 1 \}$,
    \end{enumerate}
    \item Usando el corolario \ref{corollary:desigualdad_vabsoluto}, demuestre que la función $\| \phantom{x} \|: \RR[n] \longrightarrow \RR$ es una función continua en $\RR[n]$.
    \item Demostrar que si la sucesión $\{ \mathbf{x}_n \}_{n = 1}^{\infty}$ converge a $\mathbf{x}$, entonces la sucesión $\| \mathbf{x}_n \|$ converge a $\| \mathbf{x} \|$. Es decir,
    $$\lim_{n \to \infty} \mathbf{x}_n = \mathbf{x}, \text{ entonces } \lim_{n \to \infty} \| \mathbf{x}_n \| = \| \mathbf{x} \|.$$
    \item Dada una sucesión $(\mathbf{x}_n)_{n \in \NN} \subset \RR[k]$, donde $\mathbf{x}_n = \left(x_n^1, x_n^2, \dots, x_n^k\right)$, y un vector fijo $\mathbf{x}_0 = \left(x_0^1, x_0^2, \dots, x_0^k\right)$, entonces la sucesión $(\mathbf{x}_n)_{n \in \NN}$ converge a $\mathbf{x}_0$ en $\RR[k]$ si y solo si para cada $r \in \{ 1, 2, \dots, k \}$, la sucesión de componentes $(x_n^r)_{n \in \NN}$ converge a $x_0^r$ en $\RR$.
    \item Demostrar que toda sucesión convergente en $\RR[k]$ es una sucesión de Cauchy.
    \item Demostrar que toda sucesión de Cauchy en $\RR[k]$ es una sucesión que converge en $\RR[k]$, es decir: $\RR[k]$ es un espacio de Banach.
    \item Sea $\mathbf{x} = (x_1, x_2, \dots, x_n) \in \RR[n]$, entonces
    \begin{enumerate}
        \item $|x_i| \leq \| \mathbf{x} \| \leq \sqrt{n} \sup \{ |x_1|, |x_2|, \dots, |x_n| \}$,
        \item $\| \mathbf{x} \|_{\infty} \leq \| \mathbf{x} \| \leq \sqrt{n} \| \mathbf{x} \|_{\infty}$.
    \end{enumerate}
    \item Dado $\mathbf{x} = (x_1, x_2, \dots, x_n) \in \RR[n]$. Demostrar que \label{ej:equivalencianormas}
    \begin{enumerate}
        \item $\| \mathbf{x} \|_{\infty} \leq \| \mathbf{x} \| \leq \sqrt{n} \| \mathbf{x} \|_{\infty}$ (esto demuestra que $\| \phantom{x} \|$ y $\| \phantom{x} \|_{\infty}$ son equivalentes),
        \item $\| \mathbf{x} \|_{\infty} \leq \| \mathbf{x} \|_1 \leq n \| \mathbf{x} \|_{\infty}$ (esto demuestra que $\| \phantom{x} \|_1$ y $\| \phantom{x} \|_{\infty}$ son equivalentes).
        \item Deducir que $\| \phantom{x} \|_1$ y $\| \phantom{x} \|$ son equivalentes.
    \end{enumerate}
    \item Si se define $\displaystyle \| \mathbf{x} \|_p = \left( \sum_{i=1}^n |x_i|^p \right)^{1/p}$, demuestre que \label{ej:normap}
    \begin{enumerate}
        \item $\| \mathbf{x} \|_{\infty}^p \leq \| \mathbf{x} \|_p^p$,
        \item $\| \mathbf{x} \|_p \leq n^{1/p} \| \mathbf{x} \|_{\infty}$.
        \item Deducir que $\| \phantom{x} \|_{\infty}$ y $\| \phantom{x} \|_p$ son equivalentes.
    \end{enumerate}
    \item Con los vectores de los siguientes incisos, comprobar los problemas \ref{ej:equivalencianormas} y \ref{ej:normap}.
    \begin{tasks}[label=\roman*)](2)
        \task $\mathbf{x} = (-1, -2, 5, -9) \in \RR[4]$
        \task $\mathbf{x} = (15, -2, 4) \in \RR[3]$
    \end{tasks}
\end{enumerate}


\begin{comment}
\begin{definicion}{}{}
    Por \emph{segmento} $(a, b)$ queremos significar el conjunto de todos los números reales $x$ tales que $a < x < b$. Por \emph{intervalo} $[a, b]$ entendemos el conjunto de todos los números reales $x$ tales que $a \leq x \leq b$.
\end{definicion}

Ocasionalmente encontraremos “intervalos semi-abiertos” $[a, b)$ y $(a, b]$, el primero de los cuales está constituido por todo $x$ tal que $a \leq x < b$, y el segundo por todo $x$ para el cual $a < x \leq b$. Si $a_i < b_i$, para $i = 1, 2, \dots, n$, el conjunto de todos los puntos, $\mathbf{x} = (x_1, x_2, \dots, x_n)$ en $\RR[n]$, cuyas coordenadas satisfacen las desigualdades $a_i \leq x_i \leq b_i$ ($1 \leq i \leq n$) se llama \emph{celda-$n$}. Así, una celda-$1$ es un intervalo, una celda-$2$ un rectángulo, etc.





\newpage

\begin{theorem}{}{}
    \TituloBox{Desigualdad de Cauchy-Schwarz:} Sea $V$ un espacio vectorial sobre $K$ con producto interno $\langle \, , \rangle$. Sean $\mathbf{x}$, $\mathbf{y} \in V$, entonces se cumple
    $$| \langle \mathbf{x}, \mathbf{y} \rangle | \leq \| \mathbf{x} \| \| \mathbf{y} \|.$$

    \tcblower
    \demostracion Sean $\mathbf{x}$, $\mathbf{y} \in V$, con $\mathbf{y} \neq \mathbf{0}$ y sea $\lambda \in K$. Consideremos al vector $\mathbf{x} + \lambda \mathbf{y}$, entonces se tiene
    \begin{align*}
        0 \leq \| \mathbf{x} + \lambda \mathbf{y} \|^2 & = \langle \mathbf{x} + \lambda\mathbf{y}, \mathbf{x} + \lambda\mathbf{y} \rangle \\
        & = \langle \mathbf{x}, \mathbf{x} + \lambda \mathbf{y} \rangle + \langle \lambda \mathbf{y}, \mathbf{x} + \lambda\mathbf{y} \rangle \\
        & = \langle \mathbf{x}, \mathbf{x} \rangle + \langle \mathbf{x}, \lambda \mathbf{y} \rangle + \langle \lambda \mathbf{y}, \mathbf{x} \rangle + \langle \lambda \mathbf{y}, \lambda \mathbf{y} \rangle \\
        & = \langle \mathbf{x}, \mathbf{x} \rangle + \overline{\lambda} \langle \mathbf{x}, \mathbf{y} \rangle + \lambda \langle \mathbf{y}, \mathbf{x} \rangle + \lambda \langle \mathbf{y}, \lambda \mathbf{y} \rangle \\
        & = \langle \mathbf{x}, \mathbf{x} \rangle + \overline{\lambda} \langle \mathbf{x}, \mathbf{y} \rangle + \lambda \overline{\langle \mathbf{x}, \mathbf{y} \rangle} + \lambda\overline{\lambda} \langle \mathbf{y}, \mathbf{y} \rangle \\
        & = \| \mathbf{x} \|^2 + \overline{\lambda} \langle \mathbf{x}, \mathbf{y} \rangle + \lambda \overline{\langle \mathbf{x}, \mathbf{y} \rangle} + |\lambda|^2 \| \mathbf{y} \|^2
    \end{align*}
    Así,
    $$0 \leq \| \mathbf{x} + \lambda \mathbf{y} \|^2 = \| \mathbf{x} \|^2 + \overline{\lambda} \langle \mathbf{x}, \mathbf{y} \rangle + \lambda \overline{\langle \mathbf{x}, \mathbf{y} \rangle} + |\lambda|^2 \| \mathbf{y} \|^2.$$
    Entonces
    \begin{equation}
        0 \leq \| \mathbf{x} \|^2 + \overline{\lambda} \langle \mathbf{x}, \mathbf{y} \rangle + \lambda \overline{\langle \mathbf{x}, \mathbf{y} \rangle} + |\lambda|^2 \| \mathbf{y} \|^2, \label{desigualdad-cauchy}
    \end{equation}
    lo cual se cumple para toda $\lambda \in K$, en particular para $\displaystyle \lambda = \frac{(-1)\langle \mathbf{x}, \mathbf{y} \rangle}{\| \mathbf{y} \|^2}$. Además, es evidente que $\displaystyle \overline{\lambda} = \frac{(-1)\overline{\langle \mathbf{x}, \mathbf{y} \rangle}}{\| \mathbf{y} \|^2}$. Así que,
    $$|\lambda|^2 = \frac{(-1)\langle \mathbf{x}, \mathbf{y} \rangle}{\| \mathbf{y} \|^2} \cdot \frac{(-1)\overline{\langle \mathbf{x}, \mathbf{y} \rangle}}{\| \mathbf{y} \|^2} = \frac{|\langle \mathbf{x}, \mathbf{y} \rangle|^2}{\| \mathbf{y} \|^4}.$$
    Sustituyendo $\lambda$, $\overline{\lambda}$ y $|\lambda|^2$ en la expresión \eqref{desigualdad-cauchy}, se sigue que
    \begin{align*}
        0 & \leq \| \mathbf{x} \|^2 - \frac{\overline{\langle \mathbf{x}, \mathbf{y} \rangle}}{\| \mathbf{y} \|^2} \langle \mathbf{x}, \mathbf{y} \rangle - \frac{\langle \mathbf{x}, \mathbf{y} \rangle}{\| \mathbf{y} \|^2} \overline{\langle \mathbf{x}, \mathbf{y} \rangle} + \frac{|\langle \mathbf{x}, \mathbf{y} \rangle|^2}{\| \mathbf{y} \|^4} \| \mathbf{y} \|^2 \\
        & = \| \mathbf{x} \|^2 - \frac{|\langle \mathbf{x}, \mathbf{y} \rangle|^2}{\| \mathbf{y} \|^2} - \frac{|\langle \mathbf{x}, \mathbf{y} \rangle|^2}{\| \mathbf{y} \|^2} + \frac{|\langle \mathbf{x}, \mathbf{y} \rangle|^2}{\| \mathbf{y} \|^2} \\
        & = \| \mathbf{x} \|^2 - \frac{|\langle \mathbf{x}, \mathbf{y} \rangle|^2}{\| \mathbf{y} \|^2}
    \end{align*}
    Entonces
    $$\frac{|\langle \mathbf{x}, \mathbf{y} \rangle|^2}{\| \mathbf{y} \|^2} \leq \| \mathbf{x} \|^2,$$
    de donde se sigue que
    $$|\langle \mathbf{x}, \mathbf{y} \rangle|^2 \leq \| \mathbf{x} \|^2 \| \mathbf{y} \|^2.$$
    Así, finalmente obtenemos que
    $$|\langle \mathbf{x}, \mathbf{y} \rangle| \leq \| \mathbf{x} \| \| \mathbf{y} \|.$$
\end{theorem}

\begin{corollary}{}{}
    Si $\mathbf{x}$ e $\mathbf{y}$ son linealmente dependientes, entonces
    $$|\langle \mathbf{x}, \mathbf{y} \rangle| = \| \mathbf{x} \| \| \mathbf{y} \|.$$
    Si $\mathbf{x}$ e $\mathbf{y}$ son linealmente independientes, entonces
    $$|\langle \mathbf{x}, \mathbf{y} \rangle| \leq \| \mathbf{x} \| \| \mathbf{y} \|.$$
\end{corollary}

\end{comment}